\documentclass[12pt]{article}
\usepackage{bigints}
\usepackage{graphicx}			% Use this package to include images
\usepackage{amsmath}	
\usepackage{amssymb}
\usepackage{amsfonts}
\usepackage{polynom}
\usepackage{listings}
% A library of many standard math expressions
\graphicspath{ {./Images/} }
\usepackage[margin=1in]{geometry}% Sets 1in margins. 
\newcommand{\qed}[0]{$\blacksquare$}
\usepackage{fancyhdr}			% Creates headers and footers
\usepackage{enumerate}          %These two package give custom labels to a list
\usepackage[shortlabels]{enumitem}


% Creates the header and footer. You can adjust the look and feel of these here.
\pagestyle{fancy}
\fancyhead[l]{Aditya Gupta}
\fancyhead[c]{Math 136 Homework \#8}
\fancyhead[r]{\today}
\fancyfoot[c]{\thepage}
\renewcommand{\headrulewidth}{0.2pt} %Creates a horizontal line underneath the header
\setlength{\headheight}{15pt} %Sets enough space for the header
\begin{document}
\begin{enumerate}
\item 
Let \( \mathbf{r} = x\mathbf{i} + y\mathbf{j} + z\mathbf{k} \) and \( r = \|\mathbf{r}\| = \sqrt{x^2 + y^2 + z^2} \). Let \( f \) be a scalar function of \( r \), so \( f(r) \) is a scalar field depending on \( x, y, z \) through \( r \).

We want to compute the gradient:
\[
\nabla[f(r)] = \left\langle \frac{\partial f}{\partial x}, \frac{\partial f}{\partial y}, \frac{\partial f}{\partial z} \right\rangle
\]

Using the chain rule:
\[
\frac{\partial f}{\partial x} = \frac{df}{dr} \cdot \frac{\partial r}{\partial x}
\]

Since \( r = \sqrt{x^2 + y^2 + z^2} \), we have:
\[
\frac{\partial r}{\partial x} = \frac{x}{\sqrt{x^2 + y^2 + z^2}} = \frac{x}{r}, \quad 
\frac{\partial r}{\partial y} = \frac{y}{r}, \quad 
\frac{\partial r}{\partial z} = \frac{z}{r}
\]

Therefore:
\[
\nabla[f(r)] = f'(r) \cdot \left\langle \frac{x}{r}, \frac{y}{r}, \frac{z}{r} \right\rangle = f'(r) \cdot \frac{\mathbf{r}}{r}
\]

Hence, we conclude:
\[
\boxed{\nabla[f(r)] = f'(r) \cdot \frac{\mathbf{r}}{r}} \quad \text{for } r \ne 0
\]
\item 
Let \( u = u(x, y) \) be a function with continuous second partial derivatives. We want to show that

\[
\frac{\partial^2 u}{\partial x^2} + \frac{\partial^2 u}{\partial y^2} = \frac{\partial^2 u}{\partial r^2} + \frac{1}{r} \frac{\partial u}{\partial r} + \frac{1}{r^2} \frac{\partial^2 u}{\partial \theta^2}.
\]

First, recall the transformations between Cartesian and polar coordinates:

\[
x = r \cos \theta, \quad y = r \sin \theta.
\]

We compute the partial derivatives of \( u \) using the chain rule:

\[
\frac{\partial u}{\partial x} = \frac{\partial u}{\partial r} \frac{\partial r}{\partial x} + \frac{\partial u}{\partial \theta} \frac{\partial \theta}{\partial x}, \quad
\frac{\partial u}{\partial y} = \frac{\partial u}{\partial r} \frac{\partial r}{\partial y} + \frac{\partial u}{\partial \theta} \frac{\partial \theta}{\partial y}.
\]

We compute the derivatives:

\[
\frac{\partial r}{\partial x} = \frac{x}{\sqrt{x^2 + y^2}} = \cos \theta, \quad
\frac{\partial r}{\partial y} = \frac{y}{\sqrt{x^2 + y^2}} = \sin \theta,
\]

\[
\frac{\partial \theta}{\partial x} = -\frac{y}{x^2 + y^2} = -\frac{\sin \theta}{r}, \quad
\frac{\partial \theta}{\partial y} = \frac{x}{x^2 + y^2} = \frac{\cos \theta}{r}.
\]

So,

\[
\frac{\partial u}{\partial x} = \frac{\partial u}{\partial r} \cos \theta - \frac{1}{r} \frac{\partial u}{\partial \theta} \sin \theta,
\]
\[
\frac{\partial u}{\partial y} = \frac{\partial u}{\partial r} \sin \theta + \frac{1}{r} \frac{\partial u}{\partial \theta} \cos \theta.
\]

Now compute \( \frac{\partial^2 u}{\partial x^2} + \frac{\partial^2 u}{\partial y^2} \). Applying the product and chain rules:

\[
\frac{\partial^2 u}{\partial x^2} = \frac{\partial}{\partial x} \left( \frac{\partial u}{\partial r} \cos \theta - \frac{1}{r} \frac{\partial u}{\partial \theta} \sin \theta \right),
\]

\[
\frac{\partial^2 u}{\partial y^2} = \frac{\partial}{\partial y} \left( \frac{\partial u}{\partial r} \sin \theta + \frac{1}{r} \frac{\partial u}{\partial \theta} \cos \theta \right).
\]


\[
\nabla^2 u = \frac{\partial^2 u}{\partial x^2} + \frac{\partial^2 u}{\partial y^2} = \frac{\partial^2 u}{\partial r^2} + \frac{1}{r} \frac{\partial u}{\partial r} + \frac{1}{r^2} \frac{\partial^2 u}{\partial \theta^2}.
\]

Thus, the Laplacian in polar coordinates is given by:

\[
\boxed{
\frac{\partial^2 u}{\partial x^2} + \frac{\partial^2 u}{\partial y^2}
=
\frac{\partial^2 u}{\partial r^2} + \frac{1}{r} \frac{\partial u}{\partial r} + \frac{1}{r^2} \frac{\partial^2 u}{\partial \theta^2}.
}
\]

\item Let \( x = (x_1, x_2) \in \mathbb{R}^2 \), \( h = (h_1, h_2) \in \mathbb{R}^2 \), and define the function
\[
g(t) = f(x + th) = f(x_1 + th_1, x_2 + th_2).
\]

To compute \( g'(t) \), we apply the chain rule for multivariable functions:
\[
g'(t) = \frac{d}{dt} f(x_1 + t h_1, x_2 + t h_2)
= \frac{\partial f}{\partial x_1}(x + th) \cdot h_1 + \frac{\partial f}{\partial x_2}(x + th) \cdot h_2.
\]

In vector notation, this can be written as:
\[
g'(t) = \nabla f(x + th) \cdot h.
\]

Next, to compute \( g''(t) \), we differentiate \( g'(t) \) with respect to \( t \):
\[
g''(t) = \frac{d}{dt} \left( \frac{\partial f}{\partial x_1}(x + th) \cdot h_1 + \frac{\partial f}{\partial x_2}(x + th) \cdot h_2 \right).
\]

Using the chain rule again:
\[
g''(t) = h_1 \left( \frac{\partial^2 f}{\partial x_1^2}(x + th) \cdot h_1 + \frac{\partial^2 f}{\partial x_1 \partial x_2}(x + th) \cdot h_2 \right)
+ h_2 \left( \frac{\partial^2 f}{\partial x_2 \partial x_1}(x + th) \cdot h_1 + \frac{\partial^2 f}{\partial x_2^2}(x + th) \cdot h_2 \right).
\]

Grouping terms, we get:
\[
g''(t) = h_1^2 \frac{\partial^2 f}{\partial x_1^2}(x + th) + 2h_1 h_2 \frac{\partial^2 f}{\partial x_1 \partial x_2}(x + th) + h_2^2 \frac{\partial^2 f}{\partial x_2^2}(x + th).
\]
\end{enumerate}
\end{document}
