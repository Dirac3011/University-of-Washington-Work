\documentclass[12pt]{article}
\usepackage{bigints}
\usepackage{graphicx}			% Use this package to include images
\usepackage{amsmath}	
\usepackage{amssymb}
\usepackage{amsfonts}
\usepackage{polynom}
\usepackage{listings}
% A library of many standard math expressions
\graphicspath{ {./Images/} }
\usepackage[margin=1in]{geometry}% Sets 1in margins. 
\newcommand{\qed}[0]{$\blacksquare$}
\usepackage{fancyhdr}			% Creates headers and footers
\usepackage{enumerate}          %These two package give custom labels to a list
\usepackage[shortlabels]{enumitem}


% Creates the header and footer. You can adjust the look and feel of these here.
\pagestyle{fancy}
\fancyhead[l]{Aditya Gupta}
\fancyhead[c]{Math 407 Homework \#1}
\fancyhead[r]{\today}
\fancyfoot[c]{\thepage}
\renewcommand{\headrulewidth}{0.2pt} %Creates a horizontal line underneath the header
\setlength{\headheight}{15pt} %Sets enough space for the header
\begin{document}
\begin{enumerate}
\item \begin{enumerate}
    \item To plot the feasible region, we graph the inequalities:
    \[
    \begin{aligned}
        x &\geq 0 \\
        y &\geq 0 \\
        5x + 2y &\leq 10 \\
        3x + 4y &\leq 12 \\
        x - y &\leq 1
    \end{aligned}
    \]
    The boundary lines for the feasible region are:
    \begin{align*}
        5x + 2y &= 10 \Rightarrow y = \frac{10 - 5x}{2} \\
        3x + 4y &= 12 \Rightarrow y = \frac{12 - 3x}{4} \\
        x - y &= 1 \Rightarrow y = x - 1
    \end{align*}
    The feasible region is shown in the following image:
    
    \includegraphics[width=0.5\linewidth]{Math 407/Screenshot 2025-07-02 154108.png}


    \item To solve the LP graphically, we find all corner points (vertices) of the feasible region by solving pairs of boundary equations:
    \begin{align*}
        \text{Intersection of } 5x + 2y = 10 \text{ and } 3x + 4y = 12: \\
        5x + 2y = 10 \quad &\text{(1)} \\
        3x + 4y = 12 \quad &\text{(2)}
    \end{align*}
    Multiply (1) by 2:
    \[
        10x + 4y = 20
    \]
    Subtract (2): $10x + 4y - (3x + 4y) = 20 - 12 \Rightarrow 7x = 8 \Rightarrow x = \frac{8}{7}$. Plug into (1): $5 \cdot \frac{8}{7} + 2y = 10 \Rightarrow y = \frac{15}{7}$

    So one vertex is $\left(\frac{8}{7}, \frac{15}{7}\right)$. Repeat this for other pairs:
    \begin{itemize}
        \item $5x + 2y = 10$ and $x - y = 1 \Rightarrow (x, y) = \left(\frac{12}{7}, \frac{5}{7}\right)$
        \item $3x + 4y = 12$ and $x - y = 1 \Rightarrow (x, y) = \left(\frac{16}{7}, \frac{9}{7}\right)$ (violates $5x + 2y \leq 10$)
        \item $5x + 2y = 10$ and $y = 0 \Rightarrow (2, 0)$
        \item $3x + 4y = 12$ and $y = 0 \Rightarrow (4, 0)$ (violates $5x + 2y \leq 10$)
        \item $x - y = 1$ and $y = 0 \Rightarrow (1, 0)$
    \end{itemize}

    Now evaluate the objective function $x + y$ at each valid vertex:
    \[
    \begin{aligned}
        \left(\frac{8}{7}, \frac{15}{7}\right): &\quad x + y = \frac{23}{7} \\
        \left(\frac{12}{7}, \frac{5}{7}\right): &\quad x + y = \frac{17}{7} \\
        (2, 0): &\quad x + y = 2 \\
        (1, 0): &\quad x + y = 1
    \end{aligned}
    \]

    The maximum is at $\left(\frac{8}{7}, \frac{15}{7}\right)$, with value $\frac{23}{7}$.

    \item We now replace the objective function with $x + \lambda y$. We want to find for which $\lambda \in \mathbb{R}$ the same optimal point remains optimal.

    The point $\left(\frac{8}{7}, \frac{15}{7}\right)$ lies at the intersection of:
    \[
        5x + 2y = 10 \quad \text{and} \quad 3x + 4y = 12
    \]
    Let the objective vector be $[1, \lambda]$. This point remains optimal if $[1, \lambda]$ lies in the cone generated by the gradients (normals) of the active constraints:
    \[
        \nabla_1 = [5, 2], \quad \nabla_2 = [3, 4]
    \]
    We want:
    \[
        [1, \lambda] = a[5, 2] + b[3, 4], \quad \text{for some } a, b \geq 0
    \]
    This leads to:
    \[
    \begin{cases}
        5a + 3b = 1 \\
        2a + 4b = \lambda
    \end{cases}
    \]
    Solve the first: $a = \frac{1 - 3b}{5}$

    Substitute into the second:
    \[
    \lambda = 2\left(\frac{1 - 3b}{5}\right) + 4b = \frac{2 - 6b + 20b}{5} = \frac{2 + 14b}{5}
    \]
    Require $a \geq 0 \Rightarrow \frac{1 - 3b}{5} \geq 0 \Rightarrow b \leq \frac{1}{3}$ and $b \geq 0$

    So for $b \in \left[0, \frac{1}{3}\right]$, we get:
    \[
    \lambda \in \left[\frac{2}{5}, \frac{4}{3}\right]
    \]
\end{enumerate}

\item 


To determine whether the $j$-th inequality is redundant, solve the linear program:
\[
\begin{aligned}
\max_{\mathbf{x}} \quad & \mathbf{a}_j^T \mathbf{x} \\
\text{subject to} \quad & \mathbf{a}_i^T \mathbf{x} \leq b_i, \quad \text{for all } i \in \{1,\dots,m\} \setminus \{j\}.
\end{aligned}
\]

Let $z^*$ be the optimal value of this LP. Then:

\begin{itemize}
    \item If $z^* \leq b_j$, then the constraint $\mathbf{a}_j^T \mathbf{x} \leq b_j$ is \textbf{redundant}, since it is automatically satisfied by all feasible solutions of the reduced system.
    \item If $z^* > b_j$, then there exists a feasible solution to the reduced system that violates $\mathbf{a}_j^T \mathbf{x} \leq b_j$, so the constraint is \textbf{not redundant}.
\end{itemize}

The $j$-th constraint $\mathbf{a}_j^T \mathbf{x} \leq b_j$ is \textbf{redundant} if and only if
\[
z^* = \max_{\mathbf{x}} \ \mathbf{a}_j^T \mathbf{x} \quad \text{subject to} \quad \mathbf{a}_i^T \mathbf{x} \leq b_i \ \text{for all } i \neq j,
\]
satisfies
\[
z^* \leq b_j.
\]

\item \begin{enumerate}
    \item Using the variables $\mathbf{x} = (a, b, c, d, e)^T$, where:
    \begin{itemize}
        \item $a$ = kg of apples,
        \item $b$ = kg of bananas,
        \item $c$ = kg of carrots,
        \item $d$ = kg of dates,
        \item $e$ = kg of powdered egg whites,
    \end{itemize}
    the linear programming problem can be formulated as follows:

    Minimize:
    \[
    4a + 2b + 3c + 10d + 70e
    \]

    Subject to:
    \[
    \begin{aligned}
        8c + 8d &\geq 1 &\text{(Vitamin A)} \\
        46a + 123b + 60c + 4d &\geq 30 &\text{(Vitamin C)} \\
        60a + 50b + 330c + 390d + 1040e &\geq 400 &\text{(Calcium)} \\
        24a + 46b + 28c + 80d &\geq 15 &\text{(Dietary Fiber)} \\
        3a + 7b + 9c + 25d + 80e &\geq 15 &\text{(Protein)} \\
        a, b, c, d, e &\geq 0 &\text{(Non-negativity)}
    \end{aligned}
    \]

    \item Yes, the linear program is quite feasible. This is because there are many combinations of the various kinds of foods that can satisfy the nutritional requirements. 

    \item No. All inequalities are of the kind: ``greater than or equal to,'' and the variables are guaranteed to be non-negative. This allows the variables to grow however large without violating any constraints, so the feasible region extends infinitely.

    \item Yes, the linear program is bounded. Although the region is unbounded, we are minimizing a linear cost function with all positive coefficients. As the variables increase, the cost of the whole diet increases, so there would be a minimum cost that occurs at some boundary point of the feasible region. Therefore, the optimal solution exists and the problem is bounded below.
\end{enumerate}

\end{enumerate}
\end{document}
