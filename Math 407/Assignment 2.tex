\documentclass[12pt]{article}
\usepackage{bigints}
\usepackage{graphicx}			% Use this package to include images
\usepackage{amsmath}	
\usepackage{amssymb}
\usepackage{amsfonts}
\usepackage{polynom}
\usepackage{listings}
% A library of many standard math expressions
\graphicspath{ {./Images/} }
\usepackage[margin=1in]{geometry}% Sets 1in margins. 
\newcommand{\qed}[0]{$\blacksquare$}
\usepackage{fancyhdr}			% Creates headers and footers
\usepackage{enumerate}          %These two package give custom labels to a list
\usepackage[shortlabels]{enumitem}


% Creates the header and footer. You can adjust the look and feel of these here.
\pagestyle{fancy}
\fancyhead[l]{Aditya Gupta}
\fancyhead[c]{Math 407 Homework \#2}
\fancyhead[r]{\today}
\fancyfoot[c]{\thepage}
\renewcommand{\headrulewidth}{0.2pt} %Creates a horizontal line underneath the header
\setlength{\headheight}{15pt} %Sets enough space for the header
\begin{document}
\begin{enumerate}
\item 
Let the decision variables for each week \( t = 1, 2, 3, 4 \) be defined as follows:

\begin{itemize}
  \item \( a_t \): number of assemblers in week \( t \)
  \item \( i_t \): number of instructors in week \( t \)
  \item \( e_t \): number of idle workers in week \( t \)
  \item \( w_t \): total number of paid workers in week \( t \)
  \item \( r_t \): number of radios produced in week \( t \)
  \item \( t_t \): number of trainees in week \( t \)
  \item \( x_t \): number of new trainees in week \( t \)
\end{itemize}

\textbf{Objective:}

\[
\text{Maximize } Z = 15r_1 + 13r_2 + 11r_3 + 9r_4 - \sum_{t=1}^{4}(200w_t + 100t_t)
\]

\textbf{Subject to:}

\begin{align*}
r_t &= 50a_t &&\text{for } t = 1,2,3,4 \\
w_t &= a_t + i_t + e_t &&\text{for } t = 1,2,3,4 \\
t_t &= x_t &&\text{for } t = 1,2,3 \\
x_t &\leq 3i_t &&\text{for } t = 1,2,3 \\
x_4 &= 0 \\
w_1 &= 40 \\
w_2 &= w_1 + x_1 \\
w_3 &= w_2 + x_2 \\
w_4 &= w_3 + x_3 \\
r_1 + r_2 + r_3 + r_4 &\geq 20{,}000 \\
a_t, i_t, e_t, w_t, r_t, t_t, x_t &\in \mathbb{Z}_{\geq 0} &&\text{for all } t
\end{align*}

\item 
We are given the polyhedron in \( \mathbb{R}^3 \) defined by:
\begin{align*}
1.& \; x \geq 0 \\
2.& \; y \geq 0 \\
3.& \; z \geq 0 \\
4.& \; x + 4y \leq 4 \\
5.& \; x + 2y + 3z \leq 6
\end{align*}



\begin{enumerate}[leftmargin=2cm, label=\textbf{Case \arabic*:}]

\item (1, 2, 3)  
\[
\begin{cases}
x = 0, \\ 
y = 0, \\ 
z = 0
\end{cases}
\quad 
\Longrightarrow 
\begin{cases}
x + 4y = 0 \leq 4, \\ 
x + 2y + 3z = 0 \leq 6.
\end{cases}
\quad \text{Valid.}
\]

\item (1, 2, 4)  
\[
\begin{cases}
x = 0, \\ 
y = 0, \\ 
x + 4y = 4 \Longrightarrow 0 = 4.
\end{cases}
\quad \text{No solution.}
\]

\item (1, 2, 5)  
\[
\begin{cases}
x = 0, \\ 
y = 0, \\ 
x + 2y + 3z = 6 \Longrightarrow z = 2.
\end{cases}
\quad 
\Longrightarrow 
\begin{cases}
z = 2 \geq 0, \\
x + 4y = 0 \leq 4.
\end{cases}
\quad \text{Valid.}
\]

\item (1, 3, 4)  
\[
\begin{cases}
x = 0, \\ 
z = 0, \\ 
x + 4y = 4 \Longrightarrow y = 1.
\end{cases}
\quad 
\Longrightarrow 
\begin{cases}
y = 1 \geq 0, \\
x + 2y + 3z = 2 \leq 6.
\end{cases}
\quad \text{Valid.}
\]

\item (1, 3, 5)  
\[
\begin{cases}
x = 0, \\ 
z = 0, \\ 
x + 2y + 3z = 6 \Longrightarrow y = 3.
\end{cases}
\quad 
\Longrightarrow 
x + 4y = 12 \not\leq 4.
\quad \text{Not valid.}
\]

\item (1, 4, 5)  
\[
\begin{cases}
x = 0, \\ 
x + 4y = 4 \Longrightarrow y = 1, \\ 
x + 2y + 3z = 6 \Longrightarrow z = \frac{4}{3}.
\end{cases}
\quad 
\Longrightarrow 
\begin{cases}
y = 1 \geq 0, \\
z = \frac{4}{3} \geq 0.
\end{cases}
\quad \text{Valid.}
\]

\item (2, 3, 4)  
\[
\begin{cases}
y = 0, \\ 
z = 0, \\ 
x + 4y = 4 \Longrightarrow x = 4.
\end{cases}
\quad 
\Longrightarrow 
\begin{cases}
x = 4 \geq 0, \\
x + 2y + 3z = 4 \leq 6.
\end{cases}
\quad \text{Valid.}
\]

\item (2, 3, 5)  
\[
\begin{cases}
y = 0, \\ 
z = 0, \\ 
x + 2y + 3z = 6 \Longrightarrow x = 6.
\end{cases}
\quad 
\Longrightarrow 
x + 4y = 6 \not\leq 4.
\quad \text{Not valid.}
\]

\item (2, 4, 5)  
\[
\begin{cases}
y = 0, \\ 
x + 4y = 4 \Longrightarrow x = 4, \\ 
x + 2y + 3z = 6 \Longrightarrow z = \frac{2}{3}.
\end{cases}
\quad 
\Longrightarrow 
\begin{cases}
x = 4 \geq 0, \\
z = \frac{2}{3} \geq 0.
\end{cases}
\quad \text{Valid.}
\]

\item (3, 4, 5)  
\[
\begin{cases}
z = 0, \\ 
x + 4y = 4, \\ 
x + 2y = 6.
\end{cases}
\quad 
\Longrightarrow 
2y = -2 \Longrightarrow y = -1.
\quad \text{Not valid.}
\]

\end{enumerate}
\[
\begin{aligned}
(0, 0, 0) &\leftrightarrow \{1, 2, 3\} \\
(0, 0, 2) &\leftrightarrow \{1, 2, 5\} \\
(0, 1, 0) &\leftrightarrow \{1, 3, 4\} \\
(0, 1, \tfrac{4}{3}) &\leftrightarrow \{1, 4, 5\} \\
(4, 0, 0) &\leftrightarrow \{2, 3, 4\} \\
(4, 0, \tfrac{2}{3}) &\leftrightarrow \{2, 4, 5\}
\end{aligned}
\]

\item 
\begin{enumerate}
  \item 
  We want a nonzero vector $\mathbf{w}$ such that both $\mathbf{v} + \mathbf{w}$ and $\mathbf{v} - \mathbf{w}$ remain feasible.

  The point is $\mathbf{v} = (1, 1, 1, 0, 0, 0, 2)$.  
  To keep non-negativity:
  \[
    1 \pm w_1 \ge 0,\quad 1 \pm w_2 \ge 0,\quad 1 \pm w_3 \ge 0,\quad 0 \pm w_4 \ge 0,\quad 0 \pm w_5 \ge 0,\quad 0 \pm w_6 \ge 0,\quad 2 \pm w_7 \ge 0.
  \]
  So $w_4 = w_5 = w_6 = 0$ and $w_7 = 0$.  
  Next, the constraints imply:
  \[
    (1 \pm w_1) + (1 \pm w_2) \le 2 \implies w_1 + w_2 = 0,\quad
    (1 \pm w_1) + (1 \pm w_3) \le 2 \implies w_1 + w_3 = 0.
  \]
  So $w_2 = -w_1$ and $w_3 = -w_1$.  
  Any nonzero multiple works. A valid choice is:
  \[
    \mathbf{w} = (1, -1, -1, 0, 0, 0, 0)
  \]

  \item 
  The feasible points are $\mathbf{v} + \lambda \mathbf{w}$ for $\lambda \in \mathbb{R}$:
  \[
    \mathbf{v} + \lambda \mathbf{w} = (1+\lambda,\, 1-\lambda,\, 1-\lambda,\, 0,\, 0,\, 0,\, 2).
  \]
  Non-negativity requires:
  \[
    1+\lambda \ge 0, \quad 1-\lambda \ge 0 \implies \lambda \in [-1, 1].
  \]
  The other constraints remain satisfied for these values. So,
  \[
    \lambda \in [-1, 1]
  \]

  \item 
  The objective function is:
  \[
    x_1 + x_2 + x_4 + x_7 = (1+\lambda) + (1-\lambda) + 0 + 2 = 4.
  \]
  So the objective is constant for all feasible $\lambda$. All feasible points achieve the same maximum value of 4
\end{enumerate}

\item \begin{enumerate}
\begin{itemize}
  \item Rewrite \( 2x_1 + x_2 - 20x_3 \geq -30 \) as \(-2x_1 - x_2 + 20x_3 \leq 30\).  
  \item Add slack variable \( s_1 \) for this inequality.
  \item For the lower bound \( x_3 \geq 1 \), substitute \( x_3 = 1 + x_3' \) with \( x_3' \geq 0 \).
  \item For the upper bound \( x_3 \leq 4 \): \( 1 + x_3' \leq 4 \implies x_3' \leq 3 \), so add slack \( s_2 \).
  \item For \( x_2 \leq 0 \), write \( x_2 = -x_2' \) with \( x_2' \geq 0 \).
\end{itemize}
  \item 
  \[
  \begin{cases}
  \text{maximize} & x_1 + 12x_2' - 2(1 + x_3') \\
  \text{subject to} 
  & 5x_1 + x_2' - 2(1 + x_3') = 10 \\
  & -2x_1 - x_2' + 20(1 + x_3') + s_1 = 30 \\
  & x_3' + s_2 = 3 \\
  & x_1,\, x_2',\, x_3',\, s_1,\, s_2 \geq 0.
  \end{cases}
  \]

  \item 

  \begin{itemize}
  \item Convert minimization to maximization: maximize \( -3x_1 + 12x_2 - 4x_3 \).
  \item Rewrite \( 2x_1 - x_2 - 17x_3 \geq -10 \) as \(-2x_1 + x_2 + 17x_3 \leq 10\). Add slack \( s_1 \).
  \item \( x_1 + x_2 + x_3 \leq 10 \): add slack \( s_2 \).
  \item \( x_3 \geq 1 \): substitute \( x_3 = 1 + x_3' \) with \( x_3' \geq 0 \).
  \item \( x_2 \leq 0 \): substitute \( x_2 = -x_2' \) with \( x_2' \geq 0 \).
\end{itemize}
  \[
  \begin{cases}
  \text{maximize} & -3x_1 - 12x_2' - 4(1 + x_3') \\
  \text{subject to} 
  & -2x_1 + x_2' + 17(1 + x_3') + s_1 = 10 \\
  & 5x_1 - 10(1 + x_3') = -30 \\
  & x_1 - x_2' + (1 + x_3') + s_2 = 10 \\
  & x_1,\, x_2',\, x_3',\, s_1,\, s_2 \geq 0.
  \end{cases}
  \]
\end{enumerate}

\end{enumerate}
\end{document}
