\documentclass[12pt]{article}
\usepackage{bigints}
\usepackage{graphicx}			% Use this package to include images
\usepackage{amsmath}	
\usepackage{amssymb}
\usepackage{amsfonts}
\usepackage{polynom}
\usepackage{listings}
% A library of many standard math expressions
\graphicspath{ {./Images/} }
\usepackage[margin=1in]{geometry}% Sets 1in margins. 
\newcommand{\qed}[0]{$\blacksquare$}
\usepackage{fancyhdr}			% Creates headers and footers
\usepackage{enumerate}          %These two package give custom labels to a list
\usepackage[shortlabels]{enumitem}


% Creates the header and footer. You can adjust the look and feel of these here.
\pagestyle{fancy}
\fancyhead[l]{Aditya Gupta}
\fancyhead[c]{Math 407 Homework \#3}
\fancyhead[r]{\today}
\fancyfoot[c]{\thepage}
\renewcommand{\headrulewidth}{0.2pt} %Creates a horizontal line underneath the header
\setlength{\headheight}{15pt} %Sets enough space for the header
\begin{document}
\begin{enumerate}
\item 

    \includegraphics[width=1.1\linewidth]{download.png}
\item 
\[
z = x_2 + 2x_3
\]
subject to the constraints
\[
A x = b, \quad x \geq 0
\]
where
\[
A =
\begin{pmatrix}
0 & 0 & 1 & 1 & 0 & 0 & 0 \\
-1 & 1 & -1 & 0 & 1 & 0 & 0 \\
1 & -1 & -1 & 0 & 0 & 1 & 0 \\
1 & 1 & 1 & 0 & 0 & 0 & 1
\end{pmatrix}, \quad
b =
\begin{pmatrix}
2 \\
0 \\
0 \\
3
\end{pmatrix}
\]
and the initial basis is \( B = \{4, 5, 6, 7\} \), corresponding to basic variables \( x_4, x_5, x_6, x_7 \).

Let the non-basic variables be \( x_1 = x_2 = x_3 = 0 \). Then solve:
\[
A_B x_B = b
\]
Since \( A_B = I \) (the identity matrix formed by columns 4--7), we get:
\[
x_4 = 2, \quad x_5 = 0, \quad x_6 = 0, \quad x_7 = 3
\]
The full basic feasible solution is:
\[
x = (0, 0, 0, 2, 0, 0, 3)
\]
The objective value is:
\[
z = x_2 + 2x_3 = 0
\]

We compute reduced costs:
\[
\bar{c}_j = c_j - c_B^T A_B^{-1} A_j
\]
Since \( A_B = I \), \( A_B^{-1} = I \), and \( c_B = (0, 0, 0, 0)^T \), we get:
\[
\bar{c}_j = c_j
\]
Thus:
\[
\bar{c}_1 = 0, \quad \bar{c}_2 = 1, \quad \bar{c}_3 = 2
\]
Since \( \bar{c}_3 > 0 \), variable \( x_3 \) enters the basis.



To determine the leaving variable, compute the direction vector:
\[
d = A_B^{-1} A_3 = A_3 = 
\begin{pmatrix}
1 \\ -1 \\ -1 \\ 1
\end{pmatrix}
\]
Only components where \( d_i > 0 \) are used in the ratio test:
\[
\frac{x_{B_1}}{d_1} = \frac{2}{1} = 2, \quad
\frac{x_{B_4}}{d_4} = \frac{3}{1} = 3
\]
Minimum ratio is 2, so \( x_4 \) leaves the basis.


We pivot: \( x_3 \) enters, \( x_4 \) leaves.

From the first equation:
\[
x_3 + x_4 = 2 \Rightarrow x_3 = 2 - x_4
\]

Substitute into the other constraints:

\[
- x_1 + x_2 - x_3 + x_5 &= 0 \\
\Rightarrow - x_1 + x_2 + x_5 &= x_3 = 2 - x_4 \\[1em]
\]
\[
x_1 - x_2 - x_3 + x_6 &= 0 \\
\Rightarrow x_1 - x_2 + x_6 &= 2 - x_4 \\[1em]
\]
\[
x_1 + x_2 + x_3 + x_7 &= 3 \\
\Rightarrow x_1 + x_2 + x_7 &= 1 + x_4
\]

Now let \( x_1 = x_2 = x_4 = 0 \) (non-basic). Then:
\[
x_3 = 2, \quad x_5 = 2, \quad x_6 = 2, \quad x_7 = 1
\]

Thus, the updated basic feasible solution is:
\[
x = (0, 0, 2, 0, 2, 2, 1)
\]
and the objective value is:
\[
z = x_2 + 2x_3 = 0 + 2(2) = \boxed{4}
\]

\item 
\[
\begin{aligned}
x_5 &= 1 + x_1 + x_2 + 2x_4 \\
x_6 &= 2 - x_1 + x_4 \\
x_7 &= 3 + 2x_2 + x_3 + x_4 \\
x_8 &= 5 + 2x_1 + x_2 - x_3 \\
z &= 8 + \lambda x_1 + \mu x_2 - x_3 - 2x_4
\end{aligned}
\]

We are to determine for which values of \( \lambda, \mu \in \mathbb{R} \) the linear program is \textbf{bounded}.


Set non-basic variables to zero:
\[
x_1 = x_2 = x_3 = x_4 = 0
\Rightarrow
x_5 = 1, \quad x_6 = 2, \quad x_7 = 3, \quad x_8 = 5, \quad z = 8
\]
This is a basic feasible solution.


The objective function is:
\[
z = 8 + \lambda x_1 + \mu x_2 - x_3 - 2x_4
\]
To ensure boundedness, we must verify that increasing any variable with a positive coefficient does not lead to an unbounded increase in \( z \).

\subsection*{Case 1: \( \mu > 0 \) (i.e., increase \( x_2 \))}

From the tableau:

\[
\begin{aligned}
x_5 &= 1 + x_2 \\
x_6 &= 2 \quad (\text{no } x_2) \\
x_7 &= 3 + 2x_2 \\
x_8 &= 5 + x_2
\end{aligned}
\]

All directions are nonnegative. So increasing \( x_2 \) indefinitely increases \( z \), while remaining feasible.
The LP is unbounded when \( \mu > 0 \).

\subsection*{Case 2: \( \lambda > 0 \) (i.e., increase \( x_1 \))}

From the tableau:

\[
\begin{aligned}
x_5 &= 1 + x_1 \\
x_6 &= 2 - x_1 \\
x_8 &= 5 + 2x_1
\end{aligned}
\]

\( x_6 \) decreases as \( x_1 \) increases. Minimum ratio test gives a finite bound:
\[
x_6 = 2 - x_1 \geq 0 \Rightarrow x_1 \leq 2
\]
So \( x_1 \) can only increase up to a finite bound.

The LP is still bounded for all \( \lambda \).

The linear program is bounded iff
\[
\mu \leq 0
\]
The value of \( \lambda \) does not affect boundedness.

\item 


Given that the final simplex tableau is:
\[
z = 8 - x_5 - x_6
\]

To achieve the maximum value of \( z \), we must minimize \( x_5 \) and \( x_6 \), both of which are nonnegative.

\[
\text{Thus, } z = 8 \iff x_5 = x_6 = 0
\]

\[
\boxed{x \in \mathbb{R}^6 \text{ is optimal } \iff x_5 = x_6 = 0}
\]


Given tableau:
\[
\begin{aligned}
x_2 &= 1 + x_1 - 7x_4 + 3x_5 - 2x_6 \\
x_3 &= 1 - x_1 - 5x_4 - 2x_5 + x_6 \\
z &= 8 - x_5 - x_6
\end{aligned}
\]

From part (a), set \( x_5 = x_6 = 0 \). Then:
\[
x_2 = 1 + x_1 - 7x_4, \quad x_3 = 1 - x_1 - 5x_4
\]

With nonnegativity:
\[
x_1, x_4, x_2, x_3 \geq 0
\]

We substitute and derive:

\[
\begin{cases}
1 + x_1 - 7x_4 \geq 0 \Rightarrow x_1 \geq 7x_4 - 1 \\
1 - x_1 - 5x_4 \geq 0 \Rightarrow x_1 \leq 1 - 5x_4 \\
x_1 \geq 0, \quad x_4 \geq 0
\end{cases}
\]

\[
\text{Set of optimal solutions: } \left\{
(x_1, x_4) \in \mathbb{R}_{\geq 0}^2 \;\middle|\;
7x_4 - 1 \leq x_1 \leq 1 - 5x_4
\right\}
\]

We examine the range for \( x_4 \):

\[
7x_4 - 1 \leq 1 - 5x_4 \Rightarrow 12x_4 \leq 2 \Rightarrow x_4 \leq \frac{1}{6}
\]

Also:
\[
x_1 \geq 0 \Rightarrow 7x_4 - 1 \geq 0 \Rightarrow x_4 \geq \frac{1}{7}
\]

So \( x_4 \in \left[\frac{1}{7}, \frac{1}{6}\right] \), which is a nontrivial interval.

Therefore, the optimal solution is not unique.

\end{enumerate}
\end{document}
