\documentclass[12pt]{article}
\usepackage{bigints}
\usepackage{graphicx}			% Use this package to include images
\usepackage{amsmath}	
\usepackage{amssymb}
\usepackage{amsfonts}
\usepackage{polynom}
\usepackage{listings}
% A library of many standard math expressions
\graphicspath{ {./Images/} }
\usepackage[margin=1in]{geometry}% Sets 1in margins. 
\newcommand{\qed}[0]{$\blacksquare$}
\usepackage{fancyhdr}			% Creates headers and footers
\usepackage{enumerate}          %These two package give custom labels to a list
\usepackage[shortlabels]{enumitem}


% Creates the header and footer. You can adjust the look and feel of these here.
\pagestyle{fancy}
\fancyhead[l]{Aditya Gupta}
\fancyhead[c]{Math 334 Homework \#7}
\fancyhead[r]{\today}
\fancyfoot[c]{\thepage}
\renewcommand{\headrulewidth}{0.2pt} %Creates a horizontal line underneath the header
\setlength{\headheight}{15pt} %Sets enough space for the header
\begin{document}
\begin{enumerate}
\item For the cardioid \(r = 1 + \cos\theta\), the region is traced once as
\(\theta\) runs from \(0\) to \(2\pi\). The area in polar coordinates is
\[
A = \frac12 \int_{\alpha}^{\beta} r^{2}\,d\theta
   = \frac12 \int_{0}^{2\pi} (1+\cos\theta)^{2}\,d\theta.
\]
Compute
\[
(1+\cos\theta)^{2} = 1 + 2\cos\theta + \cos^{2}\theta
= 1 + 2\cos\theta + \frac{1+\cos 2\theta}{2}
= \frac32 + 2\cos\theta + \frac12\cos 2\theta .
\]
Hence
\[
A = \frac12 \int_{0}^{2\pi} \left(\frac32 + 2\cos\theta
     + \frac12\cos 2\theta\right)\,d\theta
  = \frac12 \left[\frac32(2\pi) + 0 + 0\right]
  = \frac{3\pi}{2}.
\]

\item The region is above the \(xy\)-plane (\(z \ge 0\)), below the cone
\(z = 2 - \sqrt{x^{2}+y^{2}}\), and inside the cylinder
\((x-1)^{2} + y^{2} = 1\). Use cylindrical coordinates
\(x = r\cos\theta\), \(y = r\sin\theta\), \(z = z\). The cone becomes
\[
z = 2 - r,
\]
and the cylinder becomes
\[
(x-1)^{2} + y^{2} = 1
\quad\Longrightarrow\quad
(r\cos\theta - 1)^{2} + (r\sin\theta)^{2} = 1
\]
\[
r^{2} - 2r\cos\theta + 1 = 1
\quad\Longrightarrow\quad
r(r - 2\cos\theta) = 0.
\]
Since \(r \ge 0\), the interior of the cylinder is
\(0 \le r \le 2\cos\theta\), which requires \(\cos\theta \ge 0\), so
\(-\frac{\pi}{2} \le \theta \le \frac{\pi}{2}\).
The region is between \(z = 0\) and \(z = 2 - r\). Thus
\[
V = \int_{-\pi/2}^{\pi/2} \int_{0}^{2\cos\theta} \int_{0}^{2-r} r
    \,dz\,dr\,d\theta.
\]
First integrate in \(z\):
\[
\int_{0}^{2-r} r\,dz = r(2 - r).
\]
Then
\[
V = \int_{-\pi/2}^{\pi/2} \int_{0}^{2\cos\theta} r(2 - r)\,dr\,d\theta
  = \int_{-\pi/2}^{\pi/2}
    \left[ r^{2} - \frac{r^{3}}{3} \right]_{0}^{2\cos\theta} d\theta
\]
\[
= \int_{-\pi/2}^{\pi/2}
   \left(4\cos^{2}\theta - \frac{8}{3}\cos^{3}\theta\right) d\theta
  = 2\pi - \frac{32}{9}.
\]
So the volume is
\[
V = 2\pi - \frac{32}{9}.
\]
\item Let the ball of radius \(R\) be centered at the origin. At a point with spherical
coordinates \((r,\phi,\theta)\), the distance from the boundary is \(R-r\), so the density is
\(\rho(r)=c(R-r)\). Using spherical coordinates,
\[
m = \iiint_{\text{ball}} \rho\,dV
  = \int_{0}^{2\pi} \int_{0}^{\pi} \int_{0}^{R} c(R-r)\,r^{2}\sin\phi \,dr\,d\phi\,d\theta.
\]
First integrate in \(r\):
\[
\int_{0}^{R} (R-r)r^{2}\,dr
= \int_{0}^{R} (Rr^{2}-r^{3})\,dr
= R\frac{R^{3}}{3}-\frac{R^{4}}{4}
= \frac{R^{4}}{12}.
\]
Thus
\[
m = c\left( \int_{0}^{2\pi} d\theta \right)
      \left( \int_{0}^{\pi} \sin\phi \, d\phi \right)
      \frac{R^{4}}{12}
  = c(2\pi)(2)\frac{R^{4}}{12}
  = \frac{\pi c R^{4}}{3}.
\]

\item The region \(S\) lies in the first quadrant and is bounded by
\(xy = 1\), \(xy = 3\), \(x^{2}-y^{2}=1\), and \(x^{2}-y^{2}=4\).
Use the transformation
\[
u = xy, \qquad v = x^{2}-y^{2}.
\]
Then \(S\) maps to the rectangle
\[
T = \{(u,v): 1 \le u \le 3,\ 1 \le v \le 4\}.
\]
Compute the Jacobian:
\[
D G(x,y)
=
\begin{pmatrix}
\partial u/\partial x & \partial u/\partial y \\
\partial v/\partial x & \partial v/\partial y
\end{pmatrix}
=
\begin{pmatrix}
y & x \\
2x & -2y
\end{pmatrix},
\]
so
\[
\det D G(x,y) = y(-2y) - x(2x) = -2(x^{2}+y^{2}),
\quad
|\det D G(x,y)| = 2(x^{2}+y^{2}).
\]
Since \(du\,dv = |\det D G|\,dA\),
\[
dA = \frac{1}{|\det D G|}\,du\,dv
    = \frac{1}{2(x^{2}+y^{2})}\,du\,dv.
\]
Therefore
\[
\iint_{S} (x^{2}+y^{2})\,dA
= \iint_{T} (x^{2}+y^{2})\frac{1}{2(x^{2}+y^{2})}\,du\,dv
= \frac12 \iint_{T} du\,dv.
\]
Now
\[
\iint_{T} du\,dv
= \int_{1}^{3} \int_{1}^{4} 1\,dv\,du
= (3-1)(4-1) = 2\cdot 3 = 6,
\]
so
\[
\iint_{S} (x^{2}+y^{2})\,dA = \frac12 \cdot 6 = 3.
\]
\item
\begin{enumerate}

\item Here \(F(x,y,z) = (yz,x^{2},xz)\) and \(C\) is the line segment from
\((0,0,0)\) to \((1,1,1)\). A parametrization is
\[
\mathbf r(t) = (t,t,t), \quad 0 \le t \le 1,
\]
so
\[
\mathbf r'(t) = (1,1,1).
\]
Then
\[
F(\mathbf r(t)) = (t^{2},t^{2},t^{2}),
\]
and the line integral is
\[
\int_{C} F \cdot d\mathbf x
= \int_{0}^{1} F(\mathbf r(t)) \cdot \mathbf r'(t)\,dt
= \int_{0}^{1} (t^{2},t^{2},t^{2}) \cdot (1,1,1)\,dt
= \int_{0}^{1} 3t^{2}\,dt
= 1.
\]

\item Now \(C\) is the curve \(y=x^{2}\), \(z=x^{3}\) from \((0,0,0)\) to \((1,1,1)\).
Use the parametrization
\[
\mathbf r(t) = (t,t^{2},t^{3}), \quad 0 \le t \le 1,
\]
so
\[
\mathbf r'(t) = (1,2t,3t^{2}).
\]
Then
\[
F(\mathbf r(t)) = (yz,x^{2},xz)
= (t^{2}\cdot t^{3}, t^{2}, t\cdot t^{3})
= (t^{5}, t^{2}, t^{4}).
\]
Thus
\[
\int_{C} F \cdot d\mathbf x
= \int_{0}^{1} F(\mathbf r(t)) \cdot \mathbf r'(t)\,dt
= \int_{0}^{1} \bigl(t^{5} + 2t^{3} + 3t^{6}\bigr)\,dt
= \left[\frac{t^{6}}{6} + \frac{t^{4}}{2} + \frac{3t^{7}}{7}\right]_{0}^{1}
= \frac{1}{6} + \frac{1}{2} + \frac{3}{7}
= \frac{23}{21}.
\]
\end{enumerate}

\item Let \(F(x,y,z) = (2y - z,\; y - z,\; -x)\).  
The curve \(C\) is the intersection of the cylinder \(x^{2}+y^{2}=1\) and the plane
\(z=y\), traversed anticlockwise when viewed from the positive \(z\)-axis.

A convenient parametrization is
\[
\mathbf r(t) = (\cos t,\; \sin t,\; \sin t), \qquad 0 \le t \le 2\pi,
\]
which runs once around the circle anticlockwise in the \(xy\)-projection.
Then
\[
\mathbf r'(t) = (-\sin t,\; \cos t,\; \cos t).
\]
On \(C\),
\[
F(\mathbf r(t)) = \bigl(2\sin t - \sin t,\; \sin t - \sin t,\; -\cos t\bigr)
= (\sin t,\; 0,\; -\cos t).
\]
Hence
\[
F(\mathbf r(t)) \cdot \mathbf r'(t)
= \sin t(-\sin t) + 0\cdot \cos t + (-\cos t)\cos t
= -\sin^{2} t - \cos^{2} t = -1.
\]
Therefore
\[
\int_{C} F \cdot d\mathbf x
= \int_{0}^{2\pi} -1\,dt
= -2\pi.
\]

\item Let \(U \subset \mathbb R^{2}\) be open and connected, fix \(x_{0} \in U\), and let
\(F(x,y) = (P(x,y),Q(x,y))\) be continuous and path-independent on \(U\).  
For each \(x \in U\), define
\[
f(x) = \int_{x_{0}}^{x} F \cdot d\mathbf x,
\]
where the integral is taken along any piecewise \(C^{1}\) path from \(x_{0}\) to \(x\).
Path-independence implies that this is well defined (independent of the choice of path).

Fix \((x,y) \in U\). For \(h\) small enough so that \((x+h,y) \in U\), consider the
difference
\[
f(x+h,y) - f(x,y).
\]
By path-independence, this equals the line integral of \(F\) along any path from
\((x,y)\) to \((x+h,y)\). Choose the straight segment
\[
\gamma(t) = (t,y), \qquad x \le t \le x+h.
\]
Then \(\gamma'(t) = (1,0)\), and hence
\[
f(x+h,y) - f(x,y)
= \int_{\gamma} F \cdot d\mathbf x
= \int_{x}^{x+h} F(\gamma(t)) \cdot \gamma'(t)\,dt
= \int_{x}^{x+h} P(t,y)\,dt.
\]
Thus
\[
\frac{f(x+h,y) - f(x,y)}{h}
= \frac{1}{h} \int_{x}^{x+h} P(t,y)\,dt.
\]
Since \(P\) is continuous, the mean value theorem for integrals gives
\[
\lim_{h \to 0} \frac{1}{h} \int_{x}^{x+h} P(t,y)\,dt = P(x,y),
\]
so the partial derivative \(f_{x}(x,y)\) exists and
\[
f_{x}(x,y) = P(x,y).
\]

A completely analogous argument, using the vertical segment
\(\eta(s) = (x,s)\), \(y \le s \le y+k\), shows that for \(k\) small,
\[
f(x,y+k) - f(x,y) = \int_{y}^{y+k} Q(x,s)\,ds,
\]
and therefore
\[
f_{y}(x,y) = Q(x,y).
\]

Hence \(f\) has partial derivatives everywhere on \(U\) and
\[
\nabla f(x,y) = (f_{x}(x,y), f_{y}(x,y)) = (P(x,y),Q(x,y)) = F(x,y).
\]
Because \(P\) and \(Q\) are continuous, the partial derivatives \(f_{x}\) and \(f_{y}\)
are continuous, so \(f \in C^{1}(U)\) and \(\nabla f = F\) as required.

\end{enumerate}
\end{document}
