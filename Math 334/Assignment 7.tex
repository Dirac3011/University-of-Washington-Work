\documentclass[12pt]{article}
\usepackage{bigints}
\usepackage{graphicx}			% Use this package to include images
\usepackage{amsmath}	
\usepackage{amssymb}
\usepackage{amsfonts}
\usepackage{polynom}
\usepackage{listings}
% A library of many standard math expressions
\graphicspath{ {./Images/} }
\usepackage[margin=1in]{geometry}% Sets 1in margins. 
\newcommand{\qed}[0]{$\blacksquare$}
\usepackage{fancyhdr}			% Creates headers and footers
\usepackage{enumerate}          %These two package give custom labels to a list
\usepackage[shortlabels]{enumitem}


% Creates the header and footer. You can adjust the look and feel of these here.
\pagestyle{fancy}
\fancyhead[l]{Aditya Gupta}
\fancyhead[c]{Math 334 Homework \#7}
\fancyhead[r]{\today}
\fancyfoot[c]{\thepage}
\renewcommand{\headrulewidth}{0.2pt} %Creates a horizontal line underneath the header
\setlength{\headheight}{15pt} %Sets enough space for the header
\begin{document}
\begin{enumerate}
\item 
Given 
\[
h(x)=\int_{0}^{x}\int_{0}^{y} g(t)\,dt\,dy ,
\]
the region of integration is 
\[
D=\{(y,t):0\le t\le y\le x\}.
\]
Reversing the order: \(t\) goes \(0\le t\le x\); for each \(t\), \(y\) goes \(t\le y\le x\). Thus
\[
h(x)
= \int_{0}^{x}\int_{t}^{x} g(t)\,dy\,dt
= \int_{0}^{x} g(t)\bigl[y\bigr]_{y=t}^{y=x} dt
= \int_{0}^{x} (x-t)\,g(t)\,dt.
\]

\item 
The region \(S\) is bounded below by \(z = x^{2}+y^{2}\) and above by \(z=1\). Their intersection is the circle \(x^{2}+y^{2}=1\).

\begin{itemize}

\item[(a)] Order \(z,y,x\):
\[
\iiint_{S} f\,dV
= \int_{-1}^{1}
  \int_{-\sqrt{1-x^{2}}}^{\sqrt{1-x^{2}}}
  \int_{x^{2}+y^{2}}^{1}
  f(x,y,z)\,dz\,dy\,dx.
\]

\item[(b)] Order \(y,z,x\):
\[
\iiint_{S} f\,dV
= \int_{-1}^{1}
  \int_{x^{2}}^{1}
  \int_{-\sqrt{z-x^{2}}}^{\sqrt{z-x^{2}}}
  f(x,y,z)\,dy\,dz\,dx.
\]

\item[(c)] Order \(x,y,z\):
\[
\iiint_{S} f\,dV
= \int_{0}^{1}
  \int_{-\sqrt{z}}^{\sqrt{z}}
  \int_{-\sqrt{z-y^{2}}}^{\sqrt{z-y^{2}}}
  f(x,y,z)\,dx\,dy\,dz.
\]

\end{itemize}

\item 
Let
\[
f(x,y)=
\begin{cases}
y^{-2}, & 0<x<y<1,\\[4pt]
-x^{-2}, & 0<y<x<1,\\[4pt]
0, & \text{otherwise}.
\end{cases}
\]

\textbf{(a)}  
Along the line \(x=y/2\),
\[
f\!\left(\tfrac y2,y\right)=y^{-2}\to\infty \quad (y\to0^+),
\]
and along the line \(y=x/2\),
\[
f\!\left(x,\tfrac x2\right)=-x^{-2}\to -\infty \quad (x\to0^+).
\]
Thus \(f\) is unbounded on the bounded region \(S=[0,1]\times[0,1]\).  
A Riemann integrable function on a bounded rectangle must be bounded, so \(f\) is \emph{not} integrable on \(S\).

For fixed \(y>0\),
\[
f(x,y)=
\begin{cases}
y^{-2}, & 0<x<y,\\[4pt]
-x^{-2}, & y<x<1,
\end{cases}
\]
which is bounded and piecewise continuous with a single jump at \(x=y\). Hence it is integrable on \([0,1]\).  
Similarly, for fixed \(x>0\),
\[
f(x,y)=
\begin{cases}
-x^{-2}, & 0<y<x,\\[4pt]
y^{-2}, & x<y<1,
\end{cases}
\]
also bounded and piecewise continuous, so integrable on \([0,1]\).  

\textbf{(b)}  
First compute \(\int_{0}^{1} f(x,y)\,dx\). For fixed \(y\in(0,1]\),
\[
\int_{0}^{1} f(x,y)\,dx
= \int_{0}^{y} y^{-2}\,dx + \int_{y}^{1} (-x^{-2})\,dx.
\]
Compute each term:
\[
\int_{0}^{y} y^{-2}\,dx = \frac{1}{y},\qquad
\int_{y}^{1} (-x^{-2})\,dx = 1-\frac{1}{y}.
\]
Thus
\[
\int_{0}^{1} f(x,y)\,dx = 1.
\]
Therefore
\[
\int_{0}^{1}\!\int_{0}^{1} f(x,y)\,dx\,dy
= \int_{0}^{1} 1\,dy = 1.
\]

Now compute \(\int_{0}^{1} f(x,y)\,dy\). For fixed \(x\in(0,1]\),
\[
\int_{0}^{1} f(x,y)\,dy
= \int_{0}^{x} (-x^{-2})\,dy + \int_{x}^{1} y^{-2}\,dy.
\]
Compute:
\[
\int_{0}^{x} (-x^{-2})\,dy = -\frac{1}{x},\qquad
\int_{x}^{1} y^{-2}\,dy = -1+\frac{1}{x}.
\]
Thus
\[
\int_{0}^{1} f(x,y)\,dy = -1.
\]
Therefore
\[
\int_{0}^{1}\!\int_{0}^{1} f(x,y)\,dy\,dx
= \int_{0}^{1} (-1)\,dx = -1.
\]

Hence both iterated integrals exist but
\[
\int_{0}^{1}\!\int_{0}^{1} f(x,y)\,dx\,dy = 1,
\qquad
\int_{0}^{1}\!\int_{0}^{1} f(x,y)\,dy\,dx = -1,
\]
so they are unequal.

\item
Let $\chi_{\mathbb{Q}}$ denote the characteristic function of $\mathbb{Q}\cap[0,1]$:
\[
\chi_{\mathbb{Q}}(x)=
\begin{cases}
1, & x\in\mathbb{Q},\\
0, & x\notin\mathbb{Q}.
\end{cases}
\]
Both $\mathbb{Q}$ and its complement are dense in $[0,1]$.  
Let $P$ be any partition of $[0,1]$. Every subinterval of $P$ contains both rational and irrational numbers; hence
\[
\inf\{\chi_{\mathbb{Q}}(x): x\text{ in that subinterval}\}=0,\qquad
\sup\{\chi_{\mathbb{Q}}(x): x\text{ in that subinterval}\}=1.
\]
Thus the lower sum $L(\chi_{\mathbb{Q}},P)=0$ and the upper sum
$U(\chi_{\mathbb{Q}},P)=1$ for every partition $P$. Therefore
\[
\sup_P L(\chi_{\mathbb{Q}},P)=0\neq 1=\inf_P U(\chi_{\mathbb{Q}},P),
\]
so $\chi_{\mathbb{Q}}$ is not (Riemann) integrable on $[0,1]$.

\item
We have
\[
I=\int_{0}^{2}\int_{0}^{x^{2}}\int_{0}^{y} f(x,y,z)\,dz\,dy\,dx.
\]
The region is
\[
D=\{(x,y,z): 0\le x\le 2,\ 0\le y\le x^{2},\ 0\le z\le y\}.
\]
From $0\le z\le y\le x^{2}\le 4$ we get $0\le z\le4$, and for fixed $z$ we must have
$z\le y\le4$ and $y\le x^{2}\le 4$, so $x\in[\sqrt{y},2]$.  
With order $dx\,dy\,dz$,
\[
I=\int_{0}^{4}\int_{z}^{4}\int_{\sqrt{y}}^{2} f(x,y,z)\,dx\,dy\,dz.
\]

\item
Let $E$ be the square pyramid with base the square of vertices
$(\pm1,\pm1,0)$ and apex $(0,0,1)$.  
At height $z\in[0,1]$ the cross-section is the square
\[
-1+z\le x\le 1-z,\qquad -1+z\le y\le 1-z.
\]
Thus
\[
\iiint_{E} f\,dV
=\int_{0}^{1}\int_{-1+z}^{1-z}\int_{-1+z}^{1-z} e^{x+y}\,dy\,dx\,dz.
\]
Compute the inner integrals:
\[
\int_{-1+z}^{1-z} e^{y}\,dy = e^{1-z}-e^{-1+z},\qquad
\int_{-1+z}^{1-z} e^{x}\,dx = e^{1-z}-e^{-1+z},
\]
so the double integral at fixed $z$ is
\[
\left(e^{1-z}-e^{-1+z}\right)^{2}
= e^{2-2z}+e^{2z-2}-2.
\]
Hence
\[
\iiint_{E} f\,dV
= \int_{0}^{1}\bigl(e^{2-2z}+e^{2z-2}-2\bigr)\,dz.
\]
Now
\[
\int_{0}^{1} e^{2-2z}\,dz=\frac{1}{2}(e^{2}-1),\qquad
\int_{0}^{1} e^{2z-2}\,dz=\frac{1}{2}(1-e^{-2}),\qquad
\int_{0}^{1} 2\,dz=2,
\]
so
\[
\iiint_{E} f\,dV
=\frac{1}{2}(e^{2}-1)+\frac{1}{2}(1-e^{-2})-2
= \frac{e^{2}-e^{-2}}{2}-2.
\]


\end{enumerate}
\end{document}
