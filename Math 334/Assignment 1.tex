\documentclass[12pt]{article}
\usepackage{bigints}
\usepackage{graphicx}			% Use this package to include images
\usepackage{amsmath}	
\usepackage{amssymb}
\usepackage{amsfonts}
\usepackage{polynom}
\usepackage{listings}
% A library of many standard math expressions
\graphicspath{ {./Images/} }
\usepackage[margin=1in]{geometry}% Sets 1in margins. 
\newcommand{\qed}[0]{$\blacksquare$}
\usepackage{fancyhdr}			% Creates headers and footers
\usepackage{enumerate}          %These two package give custom labels to a list
\usepackage[shortlabels]{enumitem}


% Creates the header and footer. You can adjust the look and feel of these here.
\pagestyle{fancy}
\fancyhead[l]{Aditya Gupta}
\fancyhead[c]{Math 334 Homework \#1}
\fancyhead[r]{\today}
\fancyfoot[c]{\thepage}
\renewcommand{\headrulewidth}{0.2pt} %Creates a horizontal line underneath the header
\setlength{\headheight}{15pt} %Sets enough space for the header
\begin{document}
\begin{enumerate}
\item
Cauchy--Schwarz gives
\[
\langle a,b\rangle \le \|a\|\|b\|.
\]
Multiplying by $-2$ yields
\[
-2\langle a,b\rangle \ge -2\|a\|\|b\|.
\]
Adding $\|a\|^2+\|b\|^2$ to both sides, we obtain
\[
\|a\|^2+\|b\|^2-2\langle a,b\rangle \ge \|a\|^2+\|b\|^2-2\|a\|\|b\|.
\]
The left and right sides are $\|a-b\|^2$ and $(\|a\|-\|b\|)^2$, respectively, so
\[
\|a-b\|^2 \ge (\|a\|-\|b\|)^2.
\]
Taking nonnegative square roots gives
\[
\|a-b\| \ge \bigl|\|a\|-\|b\|\bigr|,
\]
which is the desired inequality.

\item Let \(S_n=\{1/n\}\subset\mathbb{R}\) for \(n\in\mathbb{N}\). Each \(S_n\) is closed (singletons are closed in \(\mathbb{R}\)). Consider the union
\[
S=\bigcup_{n=1}^\infty S_n=\{1,1/2,1/3,\dots\}.
\]
The point \(0\) is a limit point of \(S\) (since \(1/n\to0\)) but \(0\notin S\). Hence \(S\) is not closed. Thus \(\{S_n\}_{n=1}^\infty\) is an infinite collection of closed sets whose union is not closed.

\item Take \(S=\mathbb{Q}\subset\mathbb{R}\). Since between any two real numbers there is always an irrational number, no open interval can lie entirely inside the rationals. Therefore
\[
\operatorname{int}(S)=\varnothing.
\]

On the other hand, every real number can be approximated arbitrarily closely by rational numbers. This means that if you take any point \(x\in\mathbb{R}\) and any small interval around \(x\), that interval will contain some rational numbers. Hence the closure of \(\mathbb{Q}\) is all of \(\mathbb{R}\). Thus
\[
\operatorname{int}(\overline{S})=\operatorname{int}(\mathbb{R})=\mathbb{R}.
\]

So we have
\[
\operatorname{int}(S)=\varnothing \quad\neq\quad \mathbb{R}=\operatorname{int}(\overline{S}),
\]

\item 
Let
\[
f(x,y)=\frac{xy}{x^2+y^2}.
\]

First, \(f\) is discontinuous at \((0,0)\): along \(y=0\) we have \(f(x,0)=0\), so the path limit is \(0\); along \(y=x\) (for \(x\neq0\)) we have
\[
f(x,x)=\frac{x^2}{2x^2}=\frac12,
\]
so the path limit is \(1/2\). The two path-limits differ, hence \(f\) is not continuous at \((0,0)\).

Fix \(a\in\mathbb{R}\). Define \(g(x)=f(x,a)\).
If \(a=0\) then \(g(x)=0\) for all \(x\), so \(g\) is continuous.
If \(a\neq0\) then
\[
g(x)=\frac{ax}{x^2+a^2},
\]
and since the denominator \(x^2+a^2\) never vanishes, \(g\) is a quotient of polynomials with nonzero denominator and is therefore continuous for all \(x\).

Similarly, for \(h(y)=f(a,y)\) we have \(h(y)=0\) when \(a=0\), and \(h(y)=\dfrac{ay}{a^2+y^2}\) when \(a\neq0\), so \(h\) is continuous for every \(a\).

Hence \(f(x,a)\) and \(f(a,y)\) are continuous functions of \(x\) and \(y\) respectively for every \(a\in\mathbb{R}\); that is, \(f\) is separately continuous in \(x\) and \(y\).

\item 
Let \(\{S_j\}_{j\in J}\) be a family of open sets in \(\mathbb{R}^n\), and define
\[
S = \bigcup_{j\in J} S_j.
\]
Take any point \(x \in S\). Then \(x \in S_{j_0}\) for some \(j_0\).
Since \(S_{j_0}\) is open, there is an \(r>0\) such that
\(B_r(x) \subseteq S_{j_0}\).
Because \(S_{j_0} \subseteq S\), we have \(B_r(x) \subseteq S\).
Thus every point of \(S\) has a neighborhood inside \(S\), so \(S\) is open.

\item The taxicab norm is \(\|x\|_1=\sum_{j=1}^n|x_j|\).  

(a) For the triangle inequality note for each coordinate \(j\) that
\[
|x_j+y_j|\le |x_j|+|y_j|.
\]
Summing over \(j=1,\dots,n\) gives
\[
\|x+y\|_1=\sum_{j=1}^n|x_j+y_j|\le\sum_{j=1}^n(|x_j|+|y_j|)=\|x\|_1+\|y\|_1.
\]

(b) Let \(|x|=\sqrt{\sum_{j=1}^n x_j^2}\) denote the Euclidean norm.

First, since all \(|x_j|\ge0\),
\[
|x|^2=\sum_{j=1}^n x_j^2 \le\Big(\sum_{j=1}^n|x_j|\Big)^2=\|x\|_1^2,
\]
so \(|x|\le\|x\|_1\).

Second, write \(\|x\|_1=\sum_{j=1}^n|x_j|=\langle(|x_1|,\dots,|x_n|),(1,\dots,1)\rangle\).
Applying Cauchy--Schwarz,
\[
\|x\|_1 \le \sqrt{\sum_{j=1}^n x_j^2}\sqrt{\sum_{j=1}^n 1^2}=|x|\sqrt{n},
\]
so \(\|x\|_1\le\sqrt{n}\,|x|\). Combining the two inequalities gives
\[
|x|\le\|x\|_1\le\sqrt{n}\,|x|.
\]
\item 
Let \(S\subset\mathbb{R}\) be nonempty and bounded. Put \(M=\sup S\). For each \(k\in\mathbb{N}\) the definition of supremum yields a point \(x_k\in S\) with
\[
M-\frac{1}{k}<x_k\le M.
\]
Thus \(x_k\to M\), so every neighborhood of \(M\) meets \(S\); hence \(M\in\overline{S}\). On the other hand any neighborhood of \(M\) contains points greater than \(M\), which are not in \(S\), so every neighborhood of \(M\) also meets the complement of \(S\). Therefore \(M\) is in the closure of the complement, so \(M\notin\operatorname{int}(S)\). Being in the closure but not in the interior means \(M\in\partial S\). The same argument applied to \(m=\inf S\) shows \(m\in\partial S\). In particular the sequences \(\{x_k\}\subset S\) constructed above converge to \(\sup S\), and a similar sequence exists converging to \(\inf S\).
\end{enumerate}
\end{document}
