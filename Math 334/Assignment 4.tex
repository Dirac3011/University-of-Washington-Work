\documentclass[12pt]{article}
\usepackage{bigints}
\usepackage{graphicx}			% Use this package to include images
\usepackage{amsmath}	
\usepackage{amssymb}
\usepackage{amsfonts}
\usepackage{polynom}
\usepackage{listings}
% A library of many standard math expressions
\graphicspath{ {./Images/} }
\usepackage[margin=1in]{geometry}% Sets 1in margins. 
\newcommand{\qed}[0]{$\blacksquare$}
\usepackage{fancyhdr}			% Creates headers and footers
\usepackage{enumerate}          %These two package give custom labels to a list
\usepackage[shortlabels]{enumitem}


% Creates the header and footer. You can adjust the look and feel of these here.
\pagestyle{fancy}
\fancyhead[l]{Aditya Gupta}
\fancyhead[c]{Math 334 Homework \#4}
\fancyhead[r]{\today}
\fancyfoot[c]{\thepage}
\renewcommand{\headrulewidth}{0.2pt} %Creates a horizontal line underneath the header
\setlength{\headheight}{15pt} %Sets enough space for the header
\begin{document}
\begin{enumerate}
\item Let \(I\) be an interval and suppose \(f\) is differentiable on \(I\) with
\(f'(x)>0\) for all \(x\in I\) except at finitely many points where \(f'(x)=0\).
We prove \(f\) is strictly increasing on \(I\).

Fix \(a,b\in I\) with \(a<b\). If \(f'(x)>0\) for every \(x\in(a,b)\) then
by the Mean Value Theorem there exists \(c\in(a,b)\) with
\[
f'(c)=\frac{f(b)-f(a)}{b-a}>0,
\]
so \(f(b)-f(a)>0\), and hence \(f(a)<f(b)\).

Otherwise let \(S=\{s_1<s_2<\cdots<s_n\}\) be the (finite) set of points in \((a,b)\)
where \(f'(x)=0\). Consider the partition of \([a,b]\) by these points:
\[
[a,s_1],\ [s_1,s_2],\ \dots,\ [s_{n-1},s_n],\ [s_n,b].
\]
By construction, on each open subinterval between consecutive partition points
the derivative is strictly positive. Thus if \(x<y\) lie in the same open
subinterval, the Mean Value Theorem gives a point \(c\) between \(x\) and \(y\)
with \(f'(c)>0\), hence \(f(y)>f(x)\). Therefore \(f\) is strictly increasing
on each open subinterval of the partition.

Using these strict inequalities on the successive subintervals and continuity of
\(f\) at the partition points, we obtain the chain
\[
f(a)<f(s_1)<f(s_2)<\cdots<f(s_n)<f(b),
\]
so in particular \(f(a)<f(b)\).

Since \(a<b\) were arbitrary, \(f\) is strictly increasing on \(I\). \(\qed\)

\item For each of the following functions \(f\), we compute (i) \(\nabla f\), and (ii) 
the directional derivative of \(f\) at the point \((1,-2)\) in the direction 
\(\mathbf{u} = \left(\frac{3}{5}, \frac{4}{5}\right)\).

\textbf{(a)} \(\displaystyle f(x,y) = x^2y + \sin(\pi xy)\)

First compute the partial derivatives:
\[
f_x = 2xy + \pi y \cos(\pi xy), \quad 
f_y = x^2 + \pi x \cos(\pi xy).
\]
Hence,
\[
\nabla f = \left( 2xy + \pi y \cos(\pi xy),\; x^2 + \pi x \cos(\pi xy) \right).
\]

At the point \((1, -2)\):
\[
\nabla f(1,-2) = 
\left( 2(1)(-2) + \pi(-2)\cos(-2\pi),\; 1^2 + \pi(1)\cos(-2\pi) \right)
= (-4 - 2\pi,\, 1 + \pi).
\]

The directional derivative in the direction \(\mathbf{u}\) is
\[
D_{\mathbf{u}}f(1,-2) = \nabla f(1,-2) \cdot \mathbf{u} 
= (-4 - 2\pi)\left(\frac{3}{5}\right) + (1 + \pi)\left(\frac{4}{5}\right)
= \frac{-12 - 6\pi + 4 + 4\pi}{5} = \frac{-8 - 2\pi}{5}.
\]


\textbf{(b)} \(\displaystyle f(x,y) = e^{4x - y^2}\)

Partial derivatives:
\[
f_x = 4e^{4x - y^2}, \quad f_y = -2y e^{4x - y^2}.
\]
Hence,
\[
\nabla f = \left( 4e^{4x - y^2},\; -2y e^{4x - y^2} \right).
\]

At \((1, -2)\):
\[
\nabla f(1,-2) = \left( 4e^{4(1) - (-2)^2},\; -2(-2)e^{4 - 4} \right)
= (4e^0,\, 4e^0) = (4, 4).
\]

Directional derivative:
\[
D_{\mathbf{u}}f(1,-2) = (4,4)\cdot\left(\frac{3}{5},\frac{4}{5}\right)
= \frac{12 + 16}{5} = \frac{28}{5}.
\]


\textbf{(c)} \(\displaystyle f(x,y) = \frac{x + 2y + 4}{7x + 3y}\)

Let \(N = x + 2y + 4\) and \(D = 7x + 3y\).
Then by the quotient rule,
\[
f_x = \frac{D(1) - N(7)}{D^2} = \frac{7x + 3y - 7(x + 2y + 4)}{(7x + 3y)^2}
= \frac{-11y - 28}{(7x + 3y)^2},
\]
\[
f_y = \frac{D(2) - N(3)}{D^2} = \frac{2(7x + 3y) - 3(x + 2y + 4)}{(7x + 3y)^2}
= \frac{11x - 12}{(7x + 3y)^2}.
\]

Hence,
\[
\nabla f = \left( \frac{-11y - 28}{(7x + 3y)^2},\; \frac{11x - 12}{(7x + 3y)^2} \right).
\]

At \((1, -2)\):
\[
7x + 3y = 7(1) + 3(-2) = 1,
\]
so
\[
\nabla f(1,-2) = \left( -11(-2) - 28,\; 11(1) - 12 \right) = (22 - 28,\; -1) = (-6, -1).
\]

Directional derivative:
\[
D_{\mathbf{u}}f(1,-2) = (-6, -1)\cdot\left(\frac{3}{5}, \frac{4}{5}\right)
= \frac{-18 - 4}{5} = \frac{-22}{5}.
\]

\item 
Let \(S\subset\mathbb{R}^n\) be open and suppose each partial derivative
\(\partial_j f\) exists on \(S\) and there is a constant \(M>0\) with
\(|\partial_j f(x)|\le M\) for all \(x\in S\) and all \(j=1,\dots,n\).
We show \(f\) is continuous on \(S\).

Fix \(a=(a_1,\dots,a_n)\in S\). Since \(S\) is open there exists
\(\varepsilon>0\) such that the closed box
\[
B=\{x\in\mathbb{R}^n:\ |x_j-a_j|\le\varepsilon\ \text{for }j=1,\dots,n\}
\]
is contained in \(S\). Let \(x=(x_1,\dots,x_n)\in B\) be arbitrary and
write \(x\) as a successive change from \(a\) by defining the points
\[
a^{(0)}=a,\quad a^{(1)}=(x_1,a_2,\dots,a_n),\quad
a^{(2)}=(x_1,x_2,a_3,\dots,a_n),\ \ldots,\ 
a^{(n)}=x.
\]
Then
\[
f(x)-f(a)=\sum_{j=1}^n\bigl(f(a^{(j)})-f(a^{(j-1)})\bigr).
\]
For each fixed \(j\) consider the function of one variable
\[
g_j(t)=f\bigl(a^{(j-1)}+t(x_j-a_j)e_j\bigr), \quad t\in[0,1],
\]
where \(e_j\) is the \(j\)-th standard basis vector.  
By the one-dimensional Mean Value Theorem there exists \(t_j\in(0,1)\) with
\[
f(a^{(j)})-f(a^{(j-1)})=g_j(1)-g_j(0)=g_j'(t_j)
= \partial_j f\bigl(a^{(j-1)}+t_j(x_j-a_j)e_j\bigr)\,(x_j-a_j).
\]
Using the bound on the partial derivatives we obtain
\[
\bigl|f(a^{(j)})-f(a^{(j-1)})\bigr|
\le M\,|x_j-a_j|.
\]
Summing over \(j\) gives
\[
|f(x)-f(a)| \le \sum_{j=1}^n M\,|x_j-a_j| = M\|x-a\|_1
\le M\sqrt{n}\,\|x-a\|_2.
\]
Thus for every \(\delta>0\) we can choose \(\rho=\min\{\varepsilon,\delta/(M\sqrt{n})\}\)
so that \(\|x-a\|_2<\rho\) implies \(|f(x)-f(a)|<\delta\). Therefore \(f\)
is continuous at \(a\). Since \(a\in S\) was arbitrary, \(f\) is continuous on \(S\).
\(\square\)


\item

\begin{itemize}
\item \(w=f(x,y,t),\; x=g(y,t),\; y=h(t).\) \\
Since \(x\) and \(y\) are functions of \(t\), by the chain rule
\[
\frac{dw}{dt}=f_x\frac{dx}{dt}+f_y\frac{dy}{dt}+f_t,
\]
where the partials \(f_x,f_y,f_t\) are evaluated at \((x(t),y(t),t)\).
Now \(y'(t)=h'(t)\) and
\[
\frac{dx}{dt}=g_y(y(t),t)\,y'(t)+g_t(y(t),t)
= g_y(h(t),t)\,h'(t)+g_t(h(t),t).
\]
Hence
\[
\displaystyle \frac{dw}{dt}
= f_x\bigl(g(h(t),t),h(t),t\bigr)\bigl(g_y(h(t),t)h'(t)+g_t(h(t),t)\bigr)
\]
\[
+ f_y\bigl(g(h(t),t),h(t),t\bigr)h'(t) + f_t\bigl(g(h(t),t),h(t),t\bigr).
\]

\item \(w=f(x,u,v),\; u=g(x,y),\; v=h(x,z).\) \\
Here \(w\) depends on \(x,y,z\) through \(x,u(x,y),v(x,z)\). Using the chain rule
for partials (complete dependence on \(x\)):
\[
\boxed{\partial_x w = f_1 + f_2\,\partial_x u + f_3\,\partial_x v
= f_1(x,u,v)+ f_2(x,u,v)\,g_x(x,y) + f_3(x,u,v)\,h_x(x,z),}
\]
where \(f_1=\partial_1 f,\; f_2=\partial_2 f,\; f_3=\partial_3 f\) evaluated at \((x,u,v)\). Similarly,
\[
\boxed{\partial_y w = f_2(x,u,v)\,g_y(x,y),}
\qquad
\boxed{\partial_z w = f_3(x,u,v)\,h_z(x,z).}
\]

\item \(w=f(u),\; u=g(x,y),\; y=h(x).\) \\
Since \(y\) is a function of \(x\), \(u=u(x,y(x))\). By the chain rule
\[
\frac{dw}{dx}=f'(u)\frac{du}{dx},\qquad
\frac{du}{dx}=g_x(x,y)+g_y(x,y)\frac{dy}{dx}=g_x(x,y)+g_y(x,y)h'(x).
\]
Hence
\[
\boxed{\displaystyle \frac{dw}{dx}=f'\bigl(g(x,h(x))\bigr)\bigl(g_x(x,h(x))+g_y(x,h(x))\,h'(x)\bigr).}
\]
\end{itemize}

\item Compute \(\partial_x w\) and \(\partial_y w\) for each \(w\).

In all formulas below \(f_1,f_2,f_3\) denote \(\partial_1 f,\partial_2 f,\partial_3 f\)
evaluated at the indicated argument.

\begin{itemize}
\item \(w=f(2x-y^2,\; x\sin 3y,\; x^4).\) \\
Let \(u=2x-y^2,\; v=x\sin3y,\; s=x^4\). Then
\[
\partial_x w = f_1\cdot 2 + f_2\cdot(\sin 3y) + f_3\cdot 4x^3,
\]
\[
\partial_y w = f_1\cdot(-2y) + f_2\cdot\bigl(x\cdot 3\cos 3y\bigr) + f_3\cdot 0.
\]

\item \(w=f(e^{\,x-3y},\; \log(x^2+1),\; \sqrt{y^4+4}).\) \\
With \(u=e^{x-3y},\; v=\log(x^2+1),\; s=\sqrt{y^4+4}\),
\[
\partial_x w = f_1\cdot e^{x-3y} + f_2\cdot\frac{2x}{x^2+1} + f_3\cdot 0,
\]
\[
\partial_y w = f_1\cdot(-3e^{x-3y}) + f_2\cdot 0
+ f_3\cdot\frac{4y^3}{2\sqrt{y^4+4}}
= f_1(-3e^{x-3y}) + f_3\frac{2y^3}{\sqrt{y^4+4}}.
\]

\item \(w=\arctan\bigl(f(y^2,\;2x-y,\;-4)\bigr).\) \\
Set \(F=f(y^2,2x-y,-4)\). Then \(w=\arctan(F)\) so
\[
\partial_x w=\frac{1}{1+F^2}\,\partial_x F,
\qquad
\partial_y w=\frac{1}{1+F^2}\,\partial_y F.
\]
Now \(\partial_x F = f_1\cdot 0 + f_2\cdot 2 + f_3\cdot 0 = 2f_2,\)
and \(\partial_y F = f_1\cdot 2y + f_2\cdot(-1) = 2y f_1 - f_2.\)
Thus
\[
\boxed{\partial_x w = \frac{2f_2}{1+F^2},\qquad
\partial_y w = \frac{2y f_1 - f_2}{1+F^2},}
\]
where \(f_1,f_2,f_3\) and \(F\) are evaluated at \((y^2,2x-y,-4)\).
\end{itemize}

\item Tangent planes in \(\mathbb{R}^3\).

For each surface either use \(z=g(x,y)\) and \(z-z_0=g_x(x_0,y_0)(x-x_0)+g_y(x_0,y_0)(y-y_0)\)
or \(F(x,y,z)=0\) and \(\nabla F(x_0,y_0,z_0)\cdot((x,y,z)-(x_0,y_0,z_0))=0\).

\begin{itemize}
\item \(z=x^2-y^3,\; a=(2,-1,5).\) \\
Compute \(z_x=2x,\; z_y=-3y^2\). At \((2,-1)\): \(z_x=4,\; z_y=-3\).
Tangent plane:
\[
\boxed{\,z-5 = 4(x-2) -3(y+1)\, }.
\]

\item \(x^2+2y^2+3z^2=6,\; a=(1,1,-1).\) \\
Set \(F(x,y,z)=x^2+2y^2+3z^2-6\). Then
\(\nabla F=(2x,4y,6z)\). At \((1,1,-1)\): \(\nabla F=(2,4,-6)\).
Tangent plane:
\[
2(x-1)+4(y-1)-6(z+1)=0,
\]
or equivalently
\[
\boxed{\,z=-1+\tfrac{1}{3}(x-1)+\tfrac{2}{3}(y-1)\, }.
\]

\item \(z=\sqrt{x}+\arctan y,\; a=(9,0,3).\) \\
Compute \(z_x=\tfrac{1}{2\sqrt{x}},\; z_y=\tfrac{1}{1+y^2}\). At \((9,0)\):
\(z_x=\tfrac{1}{6},\; z_y=1\). Tangent plane:
\[
\boxed{\,z-3 = \tfrac{1}{6}(x-9) + 1\cdot(y-0)\, }.
\]

\item \(xyz^2-\log(z-1)=8,\; a=(-2,-1,2).\) \\
Let \(F(x,y,z)=xyz^2-\log(z-1)-8\). Then
\[
F_x = yz^2,\quad F_y = xz^2,\quad F_z = 2xyz - \frac{1}{z-1}.
\]
At \((-2,-1,2)\): \(F_x=-4,\; F_y=-8,\; F_z=7\). Tangent plane:
\[
-4(x+2)-8(y+1)+7(z-2)=0,
\]
or
\[
\boxed{\, -4(x+2)-8(y+1)+7(z-2)=0 \, }.
\]
\end{itemize}

\item 
\begin{itemize}

\item Let \(x(t),y(t),z(t)\) be a \(C^1\) curve and suppose
\(x^2+y z^2\) is constant along the curve. Differentiating with respect to \(t\)
and using the product rule gives
\[
\frac{d}{dt}\bigl(x^2+y z^2\bigr)=2x\frac{dx}{dt}+\frac{dy}{dt}\,z^2+2y z\frac{dz}{dt}=0.
\]
At the point \((x,y,z)=(2,2,5)\) with \(\dfrac{dx}{dt}=1\) and \(\dfrac{dy}{dt}=-1\)
we substitute to obtain
\[
2\cdot 2\cdot 1 + (-1)\cdot 5^2 + 2\cdot 2\cdot 5\cdot\frac{dz}{dt}=0,
\]
so
\[
4-25+20\frac{dz}{dt}=0 \quad\Longrightarrow\quad 20\frac{dz}{dt}=21,
\]
hence
\[
\boxed{\displaystyle \frac{dz}{dt}=\frac{21}{20}.}
\]

\item Assume \(x(t)y(t)z(t)=8\) along the curve. Differentiate:
\[
\frac{d}{dt}\bigl(xyz\bigr)=x'y z + x y' z + x y z' =0.
\]
At the time when \(x'=1,\ x=5,\ y=2\) we first compute \(z\) from \(xyz=8\):
\[
z=\frac{8}{xy}=\frac{8}{5\cdot 2}=\frac{4}{5}.
\]
Plugging into the differentiated equation gives
\[
(1)\cdot 2\cdot\frac{4}{5} + 5\cdot y'\cdot\frac{4}{5} + 5\cdot 2\cdot z' =0,
\]
which simplifies to
\[
\frac{8}{5} + 4y' + 10 z' =0.
\]
Solving for \(y'\) in terms of \(z'\) yields
\[
4y' = -10 z' - \frac{8}{5}\quad\Longrightarrow\quad
\boxed{\displaystyle y' = -\frac{5}{2}\,z' - \frac{2}{5}.}
\]

\item Let \(f:U\subset\mathbb{R}^2\to\mathbb{R}\), \(g(x,y)=(x,y,f(x,y))\)
and \(h(x,y,z)=z-f(x,y)\).
The Jacobian (the \(3\times 2\) matrix) of \(g\) is
\[
Dg(x,y)=
\begin{pmatrix}
\partial_x (x) & \partial_y (x)\\[4pt]
\partial_x (y) & \partial_y (y)\\[4pt]
\partial_x f(x,y) & \partial_y f(x,y)
\end{pmatrix}
=
\begin{pmatrix}
1 & 0\\[4pt]
0 & 1\\[4pt]
f_x(x,y) & f_y(x,y)
\end{pmatrix}.
\]
The gradient of \(h\) (viewed as a function on \(\mathbb{R}^3\)) is the \(1\times 3\)
row vector
\[
\nabla h(x,y,z) = \bigl(h_x,\,h_y,\,h_z\bigr)
= \bigl(-f_x(x,y),\ -f_y(x,y),\ 1\bigr).
\]

\item To show the range of \(Dg(x,y)\) is perpendicular to \(\nabla h\)
at the point with \(z=f(x,y)\), it suffices to check that \(\nabla h(x,y,f(x,y))\)
is orthogonal to each column of \(Dg(x,y)\). The columns of \(Dg\) are
\[
g_x=(1,0,f_x(x,y))^{T},\qquad g_y=(0,1,f_y(x,y))^{T}.
\]
Take the dot products with \(\nabla h=(-f_x,-f_y,1)\):
\[
\nabla h\cdot g_x = -f_x\cdot 1 + (-f_y)\cdot 0 + 1\cdot f_x = 0,
\]
\[
\nabla h\cdot g_y = -f_x\cdot 0 + (-f_y)\cdot 1 + 1\cdot f_y = 0.
\]
Thus both columns of \(Dg\) are orthogonal to \(\nabla h\), so every vector in the range
of \(Dg(x,y)\) is perpendicular to \(\nabla h(x,y,f(x,y))\). \(\square\)

\end{itemize}




\end{enumerate}
\end{document}
