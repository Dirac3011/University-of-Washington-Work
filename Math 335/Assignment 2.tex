\documentclass[12pt]{article}
\usepackage{bigints}
\usepackage{graphicx}			% Use this package to include images
\usepackage{amsmath}	
\usepackage{amssymb}
\usepackage{amsfonts}
\usepackage{polynom}
\usepackage{listings}
% A library of many standard math expressions
\graphicspath{ {./Images/} }
\usepackage[margin=1in]{geometry}% Sets 1in margins. 
\newcommand{\qed}[0]{$\blacksquare$}
\usepackage{fancyhdr}			% Creates headers and footers
\usepackage{enumerate}          %These two package give custom labels to a list
\usepackage[shortlabels]{enumitem}


% Creates the header and footer. You can adjust the look and feel of these here.
\pagestyle{fancy}
\fancyhead[l]{Aditya Gupta}
\fancyhead[c]{Math 335 Homework \#1}
\fancyhead[r]{\today}
\fancyfoot[c]{\thepage}
\renewcommand{\headrulewidth}{0.2pt} %Creates a horizontal line underneath the header
\setlength{\headheight}{15pt} %Sets enough space for the header
\begin{document}
\begin{enumerate}

\item 
\begin{enumerate}
\item[(a)]
Consider
\[
\sum_{n=0}^{\infty} x^n \cos(n\theta), \qquad |x|<1.
\]
We have
\[
\left| x^n \cos(n\theta) \right| \le |x|^n
\]
for all $\theta \in \mathbb{R}$. Since
\[
\sum_{n=0}^{\infty} |x|^n
\]
converges for $|x|<1$, the comparison test implies that
\[
\sum_{n=0}^{\infty} x^n \cos(n\theta)
\]
converges absolutely.

\item[(b)]
Consider
\[
\sum_{n=1}^{\infty} n^{-2} \sin(n\theta).
\]
We have
\[
\left| n^{-2} \sin(n\theta) \right| \le n^{-2}
\]
for all $\theta \in \mathbb{R}$. Since
\[
\sum_{n=1}^{\infty} n^{-2}
\]
converges, the comparison test shows that the given series converges absolutely.
\end{enumerate}

\item
Suppose $\sum a_n$ is conditionally convergent. Then
\[
\sum_{a_n>0} a_n = +\infty
\quad \text{and} \quad
\sum_{a_n<0} a_n = -\infty.
\]
We can construct a rearrangement by taking enough positive terms so that the partial sum exceeds any arbritary $M>0$, then adding one negative term, and repeating. The resulting partial sums diverge to $+\infty$.

Similarly, by taking enough negative terms to make partial sums less than $-M$ before adding one positive term, one can construct a rearrangement whose partial sums diverge to $-\infty$.

\item
Consider the rearrangement
\[
1+\frac13-\frac12+\frac15+\frac17-\frac14+\frac19+\frac1{11}-\frac16+\cdots
\]
Group terms as
\[
\left(\frac{1}{2k-1}+\frac{1}{2k+1}-\frac{1}{2k}\right), \qquad k\ge1.
\]
Then
\[
\sum_{k=1}^{\infty}
\left(\frac{1}{2k-1}+\frac{1}{2k+1}-\frac{1}{2k}\right)
=
\sum_{n=1}^{\infty}\frac{(-1)^{n-1}}{n}
+
\sum_{k=1}^{\infty}
\left(\frac{1}{2k+1}-\frac{1}{2k}\right).
\]
The first sum equals $\log 2$. The second sum equals
\[
\frac12 \sum_{n=1}^{\infty}\frac{(-1)^{n-1}}{n}
= \frac12 \log 2.
\]
Hence the sum of the rearranged series is
\[
\log 2 + \frac12 \log 2 = \frac32 \log 2.
\]



\item
Consider
\[
\sum_{n=0}^{\infty} \frac{(x+2)^n}{n^2+1}.
\]
Using the root test,
\[
\limsup_{n\to\infty} \sqrt[n]{\left|\frac{(x+2)^n}{n^2+1}\right|}
= |x+2|.
\]
Thus the series converges absolutely for $|x+2|<1$, i.e.\ $-3<x<-1$.
At $x=-3$ and $x=-1$, the series reduces to
\[
\sum \frac{(-1)^n}{n^2+1} \quad \text{and} \quad \sum \frac{1}{n^2+1},
\]
both of which converge absolutely. We know this by comparing to $\sum \frac{1}{n^2}$ . Hence the series converges absolutely for
\[
-3 \le x \le -1.
\]

\item
Consider
\[
\sum_{n=1}^{\infty} n^3(2x-1)^n.
\]
By the root test,
\[
\limsup_{n\to\infty} \sqrt[n]{|n^3(2x-1)^n|}
= |2x-1|.
\]
Thus the series converges absolutely if $|2x-1|<1$, i.e.\ $0<x<1$.
At $x=0$ or $x=1$, the terms behave like $n^3$, so the series diverges.
Hence the series converges absolutely for
\[
0<x<1.
\]

\item
Consider
\[
\sum_{n=1}^{\infty} \frac{n x^{n+2}}{5^n (n+1)^2}.
\]
Rewrite as
\[
x^2 \sum_{n=1}^{\infty} \frac{n}{(n+1)^2}\left(\frac{x}{5}\right)^n.
\]
Since $\frac{n}{(n+1)^2} \sim \frac{1}{n}$, the series behaves like
\[
\sum \frac{1}{n}\left(\frac{x}{5}\right)^n.
\]
Thus it converges absolutely for $|x|<5$.
At $x=\pm 5$, the series reduces to a harmonic-type series and diverges.
Hence convergence is absolute for
\[
|x|<5.
\]

\item
Consider
\[
\sum_{n=1}^{\infty} \frac{1}{\sqrt{n}}\left(\frac{x-1}{x+1}\right)^n.
\]
Let $r=\left|\frac{x-1}{x+1}\right|$.
If $r<1$, the series converges absolutely by comparison with a geometric series.
If $r=1$, then the terms behave like $1/\sqrt{n}$, so the series diverges.
Solving $r<1$ gives $x>0$.
Hence the series converges absolutely for
\[
x>0,
\]
and diverges for $x\le 0$.

\item
Consider
\[
\sum_{n=0}^{\infty} \frac{(-1)^n (x+1)^{2n}}{3n+2}.
\]
If $|x+1|<1$, then the series converges absolutely by comparison with a geometric series.
If $|x+1|=1$, the series becomes
\[
\sum \frac{(-1)^n}{3n+2},
\]
which converges conditionally by the alternating series test.
If $|x+1|>1$, the terms do not tend to zero.
Hence the series converges absolutely for $|x+1|<1$ and conditionally for $|x+1|=1$.

\item
Consider
\[
\sum_{n=2}^{\infty} (-1)^n \log\!\left(\frac{n+1}{n}\right).
\]
Since
\[
\log\!\left(\frac{n+1}{n}\right) \sim \frac{1}{n},
\]
the series converges by the alternating series test.
However,
\[
\sum \left|\log\!\left(\frac{n+1}{n}\right)\right|
\]
diverges by comparison with the harmonic series.
Hence the series converges conditionally.

\item
Consider
\[
\sum_{n=1}^{\infty} (-1)^{n-1} \log(n\sin n^{-1}).
\]
As $n\to\infty$,
\[
\sin n^{-1} \sim n^{-1},
\quad \text{so} \quad
n\sin n^{-1} \to 1,
\]
and
\[
\log(n\sin n^{-1}) \to 0.
\]
Moreover, the terms decrease to zero in magnitude.
Thus the series converges by the alternating series test.
Since the absolute values do not form a convergent series, the convergence is conditional.

\item
Suppose $\sum a_n$ converges. Let $p>0$. Since $n^{-p}\to0$ and is decreasing, and the partial sums of $\sum a_n$ are bounded, Dirichlet's test implies that
\[
\sum n^{-p} a_n
\]
converges.
Absolute convergence cannot be guaranteed for any $p>0$ without additional assumptions on $a_n$.

\item
For $|x|<1$ we have the geometric series
\[
\sum_{n=0}^\infty x^n = \frac{1}{1-x}.
\]
Differentiate both sides with respect to $x$:
\[
\sum_{n=1}^\infty n x^{n-1} = \frac{1}{(1-x)^2}.
\]
Multiplying by $x$ gives
\[
\sum_{n=1}^\infty n x^n = \frac{x}{(1-x)^2}.
\]
Adding $\sum_{n=0}^\infty x^n = \frac{1}{1-x}$ to both sides yields
\[
\sum_{n=0}^\infty (n+1)x^n = \frac{1}{(1-x)^2}.
\]

\item
Assume $\sum_{m,n=1}^\infty (m+n)^{-p}$ converges. Then in particular the terms must tend to zero fast enough. Group terms with $m+n=k$; there are $k-1$ such pairs, so
\[
\sum_{m,n=1}^\infty (m+n)^{-p}
= \sum_{k=2}^\infty (k-1)k^{-p}.
\]
For large $k$, $(k-1)k^{-p}\sim k^{1-p}$, so convergence implies $p>2$. Conversely, if $p>2$, then
\[
\sum_{k=2}^\infty (k-1)k^{-p} \le \sum_{k=2}^\infty k^{1-p}
\]
which converges since $1-p<-1$. Hence $\sum_{m,n=1}^\infty (m+n)^{-p}$ converges if and only if $p>2$.

\item
Here $a_{mn}=1$ if $m=n$, $a_{mn}=-1$ if $m-n=1$, and $a_{mn}=0$ otherwise. First consider
\[
\sum_{n=0}^\infty \sum_{m=0}^\infty a_{mn}.
\]
For fixed $n$, only $m=n$ and $m=n+1$ contribute, so
\[
\sum_{m=0}^\infty a_{mn} = 1-1=0,
\]
hence
\[
\sum_{n=0}^\infty \sum_{m=0}^\infty a_{mn}=0.
\]
Now reverse the order:
\[
\sum_{m=0}^\infty \sum_{n=0}^\infty a_{mn}.
\]
For $m=0$, the inner sum is $1$. For $m\ge1$, the inner sum is $1-1=0$. Thus
\[
\sum_{m=0}^\infty \sum_{n=0}^\infty a_{mn}=1.
\]
Both iterated series converge, but their sums are different.

\end{enumerate}

\end{document}
