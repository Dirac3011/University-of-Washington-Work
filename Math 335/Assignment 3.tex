\documentclass[12pt]{article}
\usepackage{bigints}
\usepackage{graphicx}			% Use this package to include images
\usepackage{amsmath}	
\usepackage{amssymb}
\usepackage{amsfonts}
\usepackage{polynom}
\usepackage{listings}
% A library of many standard math expressions
\graphicspath{ {./Images/} }
\usepackage[margin=1in]{geometry}% Sets 1in margins. 
\newcommand{\qed}[0]{$\blacksquare$}
\usepackage{fancyhdr}			% Creates headers and footers
\usepackage{enumerate}          %These two package give custom labels to a list
\usepackage[shortlabels]{enumitem}


% Creates the header and footer. You can adjust the look and feel of these here.
\pagestyle{fancy}
\fancyhead[l]{Aditya Gupta}
\fancyhead[c]{Math 335 Homework \#3}
\fancyhead[r]{\today}
\fancyfoot[c]{\thepage}
\renewcommand{\headrulewidth}{0.2pt} %Creates a horizontal line underneath the header
\setlength{\headheight}{15pt} %Sets enough space for the header
\begin{document}

\begin{enumerate}
\item
\begin{enumerate}
\item
Let \(f_k(x)=x^k\) on \([0,1]\).
For each fixed \(x\in[0,1)\), we have \(x^k\to 0\) as \(k\to\infty\), while \(f_k(1)=1\) for all \(k\).
Thus
\[
\lim_{k\to\infty} f_k(x)=
\begin{cases}
0, & 0\le x<1,\\
1, & x=1.
\end{cases}
\]
The convergence is not uniform on \([0,1]\), since
\[
\sup_{x\in[0,1]} |f_k(x)-f(x)| = 1
\]
for all \(k\).
However, on any interval \([0,a]\) with \(0<a<1\),
\[
\sup_{x\in[0,a]} x^k = a^k \to 0,
\]
so the convergence is uniform on every \([0,a]\subset[0,1)\).

\item
Let \(f_k(x)=\sin^k x\) on \([0,\pi]\).
If \(x\neq \frac{\pi}{2}\), then \(|\sin x|<1\) and hence \(\sin^k x\to 0\).
At \(x=\frac{\pi}{2}\), \(\sin^k(\pi/2)=1\) for all \(k\).
Thus
\[
\lim_{k\to\infty} f_k(x)=
\begin{cases}
0, & x\neq \frac{\pi}{2},\\
1, & x=\frac{\pi}{2}.
\end{cases}
\]
The convergence is not uniform on \([0,\pi]\), since the supremum of \(|f_k-f|\) is always \(1\).
On any closed interval not containing \(\pi/2\), say \([0,\pi/2-\delta]\cup[\pi/2+\delta,\pi]\) with \(\delta>0\), we have
\[
\sup |\sin^k x|\le (\sin(\pi/2-\delta))^k \to 0,
\]
so the convergence is uniform there.

\item
Let \(f_k(x)=kxe^{-kx}\) on \([0,\infty)\).
For each fixed \(x>0\), \(kx e^{-kx}\to 0\), and clearly \(f_k(0)=0\).
Hence \(f_k(x)\to 0\) pointwise on \([0,\infty)\).
To test uniform convergence, compute the maximum:
\[
f_k'(x)=ke^{-kx}(1-kx)=0 \quad \Rightarrow \quad x=\frac{1}{k}.
\]
Then
\[
\sup_{x\ge 0} f_k(x)=f_k(1/k)=\frac{1}{e},
\]
which does not tend to \(0\).
Thus the convergence is not uniform on \([0,\infty)\).
On any interval \([\delta,\infty)\) with \(\delta>0\),
\[
\sup_{x\ge\delta} kxe^{-kx}\le k\delta e^{-k\delta}\to 0,
\]
so the convergence is uniform there.

\item
Let \(f_k(x)=(x/k)e^{-x/k}\) on \([0,\infty)\).
For each fixed \(x\ge 0\), \(x/k\to 0\) and hence \(f_k(x)\to 0\).
Thus the pointwise limit is \(0\).
The maximum occurs when
\[
\frac{d}{dx}\left(\frac{x}{k}e^{-x/k}\right)=0
\quad\Rightarrow\quad x=k,
\]
and
\[
\sup_{x\ge 0} f_k(x)=f_k(k)=\frac{1}{e}.
\]
Since this does not tend to \(0\), the convergence is not uniform on \([0,\infty)\).
On any bounded interval \([0,M]\),
\[
\sup_{x\in[0,M]} \frac{x}{k}e^{-x/k}\le \frac{M}{k}\to 0,
\]
so the convergence is uniform on every bounded interval.
\end{enumerate}

\item
\begin{enumerate}
\item
Consider \(\sum_{n=0}^\infty e^{-nx}\).
For \(x>0\), this is a geometric series with ratio \(e^{-x}<1\), hence it converges absolutely.
On any interval \([a,\infty)\) with \(a>0\),
\[
|e^{-nx}|\le e^{-na},
\]
and \(\sum e^{-na}\) converges, so the series converges uniformly by the Weierstrass M-test.
The sum is therefore continuous on \((0,\infty)\).

\item
Consider \(\sum_{n=0}^\infty \frac{x^n}{n^2+n+1}\).
For each fixed \(x\) with \(|x|\le 1\),
\[
\left|\frac{x^n}{n^2+n+1}\right|\le \frac{1}{n^2+n+1},
\]
and \(\sum \frac{1}{n^2+n+1}\) converges.
Hence the series converges absolutely and uniformly on \([-1,1]\) by the Weierstrass M-test.
The sum is therefore continuous on \([-1,1]\).

\item
Consider \(\sum_{n=1}^\infty \frac{1}{x^2+n^2}\).
For all \(x\in\mathbb{R}\),
\[
0\le \frac{1}{x^2+n^2}\le \frac{1}{n^2},
\]
and \(\sum \frac{1}{n^2}\) converges.
Thus the series converges absolutely and uniformly on \(\mathbb{R}\).
Consequently, the sum is continuous on \(\mathbb{R}\).
\end{enumerate}


\item
Let \(f_k(x)=g(x)x^k\), where \(g\) is continuous on \([0,1]\) and \(g(1)=0\).
Since \(g\) is continuous on the compact set \([0,1]\), it is bounded, so there exists \(M>0\) such that
\[
|g(x)|\le M \quad \text{for all } x\in[0,1].
\]
Fix \(\varepsilon>0\). By continuity of \(g\) at \(x=1\) and the fact that \(g(1)=0\), there exists \(\delta\in(0,1)\) such that
\[
|g(x)|<\varepsilon \quad \text{whenever } x\in(1-\delta,1].
\]
Split the interval as \([0,1-\delta]\cup(1-\delta,1]\).
On \([0,1-\delta]\),
\[
|f_k(x)| \le M(1-\delta)^k \to 0 \quad \text{as } k\to\infty,
\]
so there exists \(K_1\) such that \(M(1-\delta)^k<\varepsilon\) for all \(k\ge K_1\).
On \((1-\delta,1]\),
\[
|f_k(x)| \le |g(x)|\,|x^k| \le \varepsilon.
\]
Hence for all \(k\ge K_1\),
\[
\sup_{x\in[0,1]} |f_k(x)| \le \varepsilon,
\]
which shows that \(f_k\to 0\) uniformly on \([0,1]\).

\item
Consider the series
\[
\sum_{n=1}^\infty \frac{(-1)^{n-1}}{x^2+n}.
\]
For each fixed \(x\in\mathbb{R}\), the sequence \(a_n(x)=\frac{1}{x^2+n}\) is positive, decreases monotonically in \(n\), and satisfies \(a_n(x)\to 0\).
Hence, by the alternating series test, the series converges for every \(x\in\mathbb{R}\).
To show uniform convergence, i will let
\[
R_N(x)=\sum_{n=N+1}^\infty \frac{(-1)^{n-1}}{x^2+n}.
\]
\[
|R_N(x)| \le \frac{1}{x^2+N+1} \le \frac{1}{N+1},
\]
for all \(x\in\mathbb{R}\).
Since \(\frac{1}{N+1}\to 0\) as \(N\to\infty\), the convergence is uniform on \(\mathbb{R}\).
To show that the convergence is not absolute, we fix \(x\in\mathbb{R}\) and consider the series of absolute values
\[
\sum_{n=1}^\infty \left|\frac{(-1)^{n-1}}{x^2+n}\right|
= \sum_{n=1}^\infty \frac{1}{x^2+n}.
\]
Since \(x^2\ge 0\), choose \(N\in\mathbb{N}\) such that \(n\ge x^2\) for all \(n\ge N\).
Then for all \(n\ge N\),
\[
x^2+n \le n+n = 2n,
\]
and hence
\[
\frac{1}{x^2+n} \ge \frac{1}{2n}.
\]
Therefore,
\[
\sum_{n=N}^\infty \frac{1}{x^2+n}
\ge
\sum_{n=N}^\infty \frac{1}{2n}
=
\frac{1}{2}\sum_{n=N}^\infty \frac{1}{n}.
\]
Since the harmonic series diverges, its tail \(\sum_{n=N}^\infty \frac{1}{n}\) also diverges.
Thus \(\sum_{n=1}^\infty \frac{1}{x^2+n}\) diverges for every fixed \(x\in\mathbb{R}\).

Because the original series converges by the alternating series test, but the series of absolute values diverges, the convergence is conditional at every point.


\item
Let \(\{f_k\}\) be a sequence of functions on a set \(S\), and let \(S_1,\dots,S_M\) be subsets of \(S\) such that \(f_k\to f\) uniformly on each \(S_m\).
Fix \(\varepsilon>0\).
For each \(m=1,\dots,M\), there exists \(N_m\) such that
\[
|f_k(x)-f(x)|<\varepsilon \quad \text{for all } x\in S_m \text{ and all } k\ge N_m.
\]
Let \(N=\max\{N_1,\dots,N_M\}\).
Then for all \(k\ge N\) and all \(x\in \bigcup_{m=1}^M S_m\), there exists some \(m\) with \(x\in S_m\), and hence
\[
|f_k(x)-f(x)|<\varepsilon.
\]
Therefore \(f_k\to f\) uniformly on \(\bigcup_{m=1}^M S_m\).
\item
Let
\[
f(x)=\sum_{n=1}^\infty (x+n)^{-2}, \qquad x\in[0,\infty).
\]
For each \(x\ge 0\), the series converges by comparison with \(\sum n^{-2}\).
Moreover, for \(x\ge 0\),
\[
0 \le (x+n)^{-2} \le n^{-2},
\]
and since \(\sum_{n=1}^\infty n^{-2}\) converges, the series converges uniformly on \([0,\infty)\) by the Weierstrass M-test.
Each term is continuous, hence \(f\) is continuous on \([0,\infty)\).

To compute the integral, use monotone convergence (or term-by-term integration justified by uniform convergence on compact intervals):
\[
\int_0^1 f(x)\,dx
= \sum_{n=1}^\infty \int_0^1 (x+n)^{-2}\,dx
= \sum_{n=1}^\infty \left[ -(x+n)^{-1} \right]_0^1
= \sum_{n=1}^\infty \left( \frac{1}{n} - \frac{1}{n+1} \right).
\]
This telescopes, giving
\[
\int_0^1 f(x)\,dx = 1.
\]

\item
Let \(f_k(x)=x\arctan(kx)\).

\begin{enumerate}
\item
Fix \(x\in\mathbb{R}\).
If \(x>0\), then \(kx\to\infty\) and \(\arctan(kx)\to \frac{\pi}{2}\), so
\[
f_k(x)\to \frac{\pi}{2}x.
\]
If \(x<0\), then \(kx\to -\infty\) and \(\arctan(kx)\to -\frac{\pi}{2}\), so
\[
f_k(x)\to \frac{\pi}{2}|x|.
\]
If \(x=0\), then \(f_k(0)=0\).
Hence
\[
\lim_{k\to\infty} f_k(x)=\frac{\pi}{2}|x|.
\]

\item
For each \(k\),
\[
f_k'(x)=\arctan(kx)+\frac{kx}{1+k^2x^2}.
\]
Fix \(x\neq 0\). Then \(\arctan(kx)\to \frac{\pi}{2}\operatorname{sgn}(x)\) and \(\frac{kx}{1+k^2x^2}\to 0\), so
\[
\lim_{k\to\infty} f_k'(x)=\frac{\pi}{2}\operatorname{sgn}(x).
\]
At \(x=0\),
\[
f_k'(0)=0,
\]
so the limit exists for all \(x\).
However, the limit function is discontinuous at \(x=0\), while each \(f_k'\) is continuous.
Therefore the convergence of \(f_k'\) cannot be uniform on any interval containing \(0\).
\end{enumerate}

\item
\begin{enumerate}
\item
Consider the series
\[
\sum_{n=0}^\infty e^{-nx}, \qquad x>0.
\]
Each term is continuously differentiable and
\[
\frac{d}{dx}\bigl(e^{-nx}\bigr) = -n e^{-nx}.
\]
Fix \(a>0\) and consider \(x\in[a,\infty)\).
Then
\[
|n e^{-nx}| \le n e^{-na}.
\]
Since
\[
\sum_{n=0}^\infty n e^{-na}
\]
converges, the Weierstrass M-test implies that the series of derivatives
\[
\sum_{n=0}^\infty (-n e^{-nx})
\]
converges uniformly on \([a,\infty)\).
Moreover, the original series converges at some point \(x=a\).
Therefore, by the term-by-term differentiation theorem, the series
\[
\sum_{n=0}^\infty e^{-nx}
\]
may be differentiated term-by-term on \((0,\infty)\).

\item
Consider the series
\[
\sum_{n=0}^\infty \frac{x^n}{n^2+n+1}, \qquad |x|<1.
\]
Each term is continuously differentiable, with
\[
\frac{d}{dx}\left(\frac{x^n}{n^2+n+1}\right)
= \frac{n x^{n-1}}{n^2+n+1}.
\]
Fix \(r\) with \(0<r<1\) and restrict to \(|x|\le r\).
Then
\[
\left|\frac{n x^{n-1}}{n^2+n+1}\right|
\le \frac{n r^{n-1}}{n^2+n+1}
\le r^{n-1}.
\]
Since the geometric series \(\sum r^{n-1}\) converges, the series of derivatives converges uniformly on \([-r,r]\) by the Weierstrass M-test.
The original series converges at \(x=0\).
Hence the series may be differentiated term-by-term on \((-1,1)\).

\item
Consider the series
\[
\sum_{n=1}^\infty \frac{1}{x^2+n^2}, \qquad x\in\mathbb{R}.
\]
Each term is continuously differentiable and
\[
\frac{d}{dx}\left(\frac{1}{x^2+n^2}\right)
= -\frac{2x}{(x^2+n^2)^2}.
\]
For all \(x\in\mathbb{R}\),
\[
\left|\frac{2x}{(x^2+n^2)^2}\right|
\le \frac{2|x|}{n^4}
\le \frac{2}{n^3}.
\]
Since
\[
\sum_{n=1}^\infty \frac{1}{n^3}
\]
converges, the Weierstrass M-test shows that the series of derivatives converges uniformly on \(\mathbb{R}\).
The original series converges at \(x=0\).
Therefore the series may be differentiated term-by-term on \(\mathbb{R}\).
\end{enumerate}

\item
Let \(f\) be continuous on \([0,\infty)\) and satisfy
\[
0\le f(x)\le Cx^{-1-\varepsilon}, \qquad x>0,
\]
for some \(C,\varepsilon>0\), and define
\[
f_k(x)=k f(kx).
\]

\begin{enumerate}
\item
Fix \(x>0\). Then
\[
0\le f_k(x)=k f(kx)\le Ck(kx)^{-1-\varepsilon}
= Ck^{-\varepsilon}x^{-1-\varepsilon}\to 0.
\]
For any \(\delta>0\) and \(x\ge \delta\),
\[
f_k(x)\le Ck^{-\varepsilon}\delta^{-1-\varepsilon},
\]
which is independent of \(x\) and tends to \(0\).
Thus the convergence is uniform on \([\delta,\infty)\).

\item
By the change of variables \(u=kx\),
\[
\int_0^1 f_k(x)\,dx
= \int_0^1 kf(kx)\,dx
= \int_0^k f(u)\,du.
\]
As \(k\to\infty\), the right-hand side increases to \(\int_0^\infty f(u)\,du=a\).
Hence
\[
\lim_{k\to\infty}\int_0^1 f_k(x)\,dx=a.
\]

\item
Let \(g\) be integrable on \([0,1]\) and continuous at \(0\).
Fix \(\delta>0\) and write
\[
\int_0^1 f_k(x)g(x)\,dx
= \int_0^\delta f_k(x)g(x)\,dx + \int_\delta^1 f_k(x)g(x)\,dx.
\]
The second integral tends to \(0\) by uniform convergence of \(f_k\) on \([\delta,1]\).
On \([0,\delta]\), continuity of \(g\) at \(0\) implies \(g(x)\to g(0)\), and using part (b),
\[
\lim_{k\to\infty}\int_0^\delta f_k(x)g(x)\,dx
= g(0)\int_0^\infty f(u)\,du
= ag(0).
\]
Thus
\[
\lim_{k\to\infty}\int_0^1 f_k(x)g(x)\,dx=ag(0).
\]
\end{enumerate}

\end{enumerate}


\end{document}
