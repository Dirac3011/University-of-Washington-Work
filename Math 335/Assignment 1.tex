\documentclass[12pt]{article}
\usepackage{bigints}
\usepackage{graphicx}			% Use this package to include images
\usepackage{amsmath}	
\usepackage{amssymb}
\usepackage{amsfonts}
\usepackage{polynom}
\usepackage{listings}
% A library of many standard math expressions
\graphicspath{ {./Images/} }
\usepackage[margin=1in]{geometry}% Sets 1in margins. 
\newcommand{\qed}[0]{$\blacksquare$}
\usepackage{fancyhdr}			% Creates headers and footers
\usepackage{enumerate}          %These two package give custom labels to a list
\usepackage[shortlabels]{enumitem}


% Creates the header and footer. You can adjust the look and feel of these here.
\pagestyle{fancy}
\fancyhead[l]{Aditya Gupta}
\fancyhead[c]{Math 335 Homework \#1}
\fancyhead[r]{\today}
\fancyfoot[c]{\thepage}
\renewcommand{\headrulewidth}{0.2pt} %Creates a horizontal line underneath the header
\setlength{\headheight}{15pt} %Sets enough space for the header
\begin{document}
\begin{enumerate}
\item The series is
\[
10x^{-2}+20x^{-4}+40x^{-6}+\cdots = \sum_{n=1}^{\infty} 10\cdot 2^{n-1} x^{-2n}.
\]
This is a geometric series with ratio \(r=2/x^{2}\). It converges if \(|2/x^{2}|<1\), i.e. \(|x|>\sqrt{2}\). For \(|x|>\sqrt{2}\),
\[
\sum_{n=1}^{\infty} 10\cdot 2^{n-1} x^{-2n}
= \frac{10x^{-2}}{1-2/x^{2}}
= \frac{10}{x^{2}-2}.
\]

\item The series is
\[
1+\frac{1-x}{1+x}+\frac{(1-x)^2}{(1+x)^2}+\cdots
= \sum_{n=0}^{\infty} \left(\frac{1-x}{1+x}\right)^n.
\]
This is geometric with ratio \(r=(1-x)/(1+x)\). It converges if \(|(1-x)/(1+x)|<1\), which is equivalent to \(x>0\). For \(x>0\),
\[
\sum_{n=0}^{\infty} \left(\frac{1-x}{1+x}\right)^n
= \frac{1}{1-\frac{1-x}{1+x}}
= \frac{1+x}{2x}.
\]

\item The series is
\[
1+\frac34+\frac58+\frac9{16}+\frac{17}{32}+\cdots.
\]
The general term can be written as
\[
\frac{2^n+1}{2^{n+1}}=\frac12+\frac{1}{2^{n+1}}, \qquad n\ge0.
\]
Thus the series is
\[
\sum_{n=0}^{\infty}\left(\frac12+\frac{1}{2^{n+1}}\right),
\]
which diverges because \(\sum \frac12\) diverges. Hence the given series diverges.

\item The series is
\[
\frac{1}{1\cdot2}+\frac{1}{2\cdot3}+\frac{1}{3\cdot4}+\cdots
= \sum_{n=1}^{\infty} \frac{1}{n(n+1)}.
\]
Using partial fractions,
\[
\frac{1}{n(n+1)}=\frac{1}{n}-\frac{1}{n+1}.
\]
The partial sums telescope:
\[
\sum_{n=1}^{N}\left(\frac{1}{n}-\frac{1}{n+1}\right)=1-\frac{1}{N+1}.
\]
Letting \(N\to\infty\), the series converges to \(1\).

\item Suppose the infinite product \(\prod_{n=1}^{\infty} a_n\) converges to a nonzero number \(P\). Let \(P_k=\prod_{n=1}^{k} a_n\). Then \(P_k\to P\neq0\). Since
\[
a_k=\frac{P_k}{P_{k-1}},
\]
and both \(P_k\) and \(P_{k-1}\) tend to \(P\), it follows that
\[
\lim_{k\to\infty} a_k=\frac{P}{P}=1.
\]

\item
\[
\sum_{n=0}^{\infty} \frac{\sqrt{n+1}}{n^2-4n+5}
\]
For large $n$,
\[
\frac{\sqrt{n+1}}{n^2-4n+5}\sim \frac{n^{1/2}}{n^2}=\frac{1}{n^{3/2}}.
\]
Since $\sum 1/n^{3/2}$ converges, the given series converges by limit comparison.

\item
\[
\sum_{n=1}^{\infty} ne^{-n}
\]
Apply the ratio test:
\[
\lim_{n\to\infty}\frac{(n+1)e^{-(n+1)}}{ne^{-n}}
=\lim_{n\to\infty}\frac{n+1}{n}\cdot\frac{1}{e}
=\frac{1}{e}<1.
\]
Hence the series converges.

\item
\[
\sum_{n=0}^{\infty}\frac{1^2\cdot3^2\cdots(2n+1)^2}{3^n(2n)!}
\]
Using the ratio test,
\[
\frac{a_{n+1}}{a_n}
=\frac{(2n+3)^2}{3(2n+2)(2n+1)}.
\]
Taking limits,
\[
\lim_{n\to\infty}\frac{(2n+3)^2}{3(2n+2)(2n+1)}=\frac{1}{3}<1.
\]
Therefore the series converges.

\item
\[
\sum_{n=1}^{\infty}\frac{n!}{10^n}
\]
Apply the ratio test:
\[
\frac{a_{n+1}}{a_n}=\frac{n+1}{10}.
\]
Since this tends to $\infty$, the series diverges.

\item
\[
\sum_{n=0}^{\infty}\frac{3^n n!}{n^n}
\]
Using the ratio test,
\[
\frac{a_{n+1}}{a_n}
=3\left(\frac{n}{n+1}\right)^n.
\]
As $n\to\infty$, this tends to $3/e>1$, so the series diverges.

\item
\[
\sum_{n=1}^{\infty}\left(\frac{n}{n+1}\right)^{n^2}
\]
We have
\[
\ln\left(\frac{n}{n+1}\right)\sim -\frac{1}{n},
\quad
n^2\ln\left(\frac{n}{n+1}\right)\sim -n.
\]
Thus
\[
\left(\frac{n}{n+1}\right)^{n^2}\sim e^{-n}.
\]
Since $\sum e^{-n}$ converges, the series converges by comparison.

\item
\[
\sum_{n=1}^{\infty}\left[1-\cos\left(\frac{1}{n}\right)\right]
\]
Using the Taylor expansion,
\[
1-\cos x\sim \frac{x^2}{2},
\]
so
\[
1-\cos\left(\frac{1}{n}\right)\sim \frac{1}{2n^2}.
\]
Since $\sum 1/n^2$ converges, the series converges.

\item
\[
\sum_{n=2}^{\infty}\frac{1}{n(\log n)^p}
\]
Using the integral test,
\[
\int_2^\infty \frac{dx}{x(\log x)^p}
=\int_{\log 2}^\infty \frac{du}{u^p}.
\]
This integral converges if $p>1$ and diverges if $p\le1$.
Hence the series converges exactly when $p>1$.

\item
Using integral bounds for decreasing functions,
\[
\int_2^{10^{40}}\frac{dx}{x\log x}
<\sum_{n=2}^{10^{40}}\frac{1}{n\log n}
<1+\int_2^{10^{40}}\frac{dx}{x\log x}.
\]
Evaluating,
\[
4.88<\sum_{n=2}^{10^{40}}\frac{1}{n\log n}<5.61.
\]
Similarly,
\[
\int_{10^{40}}^\infty\frac{dx}{x(\log x)^2}
=\frac{1}{\log(10^{40})}\approx 0.011,
\]
so
\[
\sum_{n=10^{40}}^\infty\frac{1}{n(\log n)^2}\approx 0.011.
\]
\end{enumerate}
\end{document}
