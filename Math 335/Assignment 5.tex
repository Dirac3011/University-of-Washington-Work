\documentclass[12pt]{article}
\usepackage{bigints}
\usepackage{graphicx}			% Use this package to include images
\usepackage{amsmath}	
\usepackage{amssymb}
\usepackage{amsfonts}
\usepackage{polynom}
\usepackage{listings}
% A library of many standard math expressions
\graphicspath{ {./Images/} }
\usepackage[margin=1in]{geometry}% Sets 1in margins. 
\newcommand{\qed}[0]{$\blacksquare$}
\usepackage{fancyhdr}			% Creates headers and footers
\usepackage{enumerate}          %These two package give custom labels to a list
\usepackage[shortlabels]{enumitem}


% Creates the header and footer. You can adjust the look and feel of these here.
\pagestyle{fancy}
\fancyhead[l]{Aditya Gupta}
\fancyhead[c]{Math 335 Homework \#4}
\fancyhead[r]{\today}
\fancyfoot[c]{\thepage}
\renewcommand{\headrulewidth}{0.2pt} %Creates a horizontal line underneath the header
\setlength{\headheight}{15pt} %Sets enough space for the header
\begin{document}
\begin{enumerate}
    \item % Exercise 1
   We want to show that $\Gamma(2x) = \frac{2^{2x-1}}{\sqrt{\pi}} \Gamma(x) \Gamma\left(x + \frac{1}{2}\right)$ for a positive integer $x$.
    \begin{itemize}
        \item $\Gamma(x) = (x-1)! = (x-1)(x-2)\dots(2)(1)$
        \item $\Gamma\left(x + \frac{1}{2}\right) = \left(x - \frac{1}{2}\right)\left(x - \frac{3}{2}\right)\dots\left(\frac{3}{2}\right)\left(\frac{1}{2}\right)\sqrt{\pi}$
    \end{itemize}

    Substituting these into the RHS:
    $$RHS = \frac{2^{2x-1}}{\sqrt{\pi}} \cdot [(x-1)(x-2)\dots(1)] \cdot \left[\left(x - \frac{1}{2}\right)\left(x - \frac{3}{2}\right)\dots\left(\frac{1}{2}\right)\sqrt{\pi}\right]$$

    Cancel $\sqrt{\pi}$ and split the $2^{2x-1}$ term into $2^{x-1} \cdot 2^x$:
    $$RHS = 2^{x-1} \cdot [(x-1)(x-2)\dots(1)] \cdot 2^x \cdot \left[\left(x - \frac{1}{2}\right)\left(x - \frac{3}{2}\right)\dots\left(\frac{1}{2}\right)\right]$$

    Now, distribute $2^{x-1}$ into the first factorial term and $2^x$ into the second (which has $x$ terms):
    \begin{itemize}
        \item Distributed $2^{x-1}$: $[(2x-2)(2x-4)\dots(4)(2)]$
        \item Distributed $2^x$: $[(2x-1)(2x-3)\dots(3)(1)]$
    \end{itemize}

    Combining these two sets of terms together in descending order:
    $$RHS = (2x-1) \cdot (2x-2) \cdot (2x-3) \cdot (2x-4) \dots 4 \cdot 3 \cdot 2 \cdot 1$$
    
    This is exactly the definition of $(2x-1)!$. Since $\Gamma(2x) = (2x-1)!$, the identity is proven.
    \item % Exercise 3
    \begin{enumerate}
        \item % 3a
        Let $u = x^2$, then $du = 2x \, dx$, or $dx = \frac{1}{2}u^{-1/2} du$. \\
        Substituting into the integral:
        $$I = \int_{0}^{\infty} (u^2) e^{-u} \frac{1}{2}u^{-1/2} du = \frac{1}{2} \int_{0}^{\infty} u^{3/2} e^{-u} du$$
        This matches the Gamma function form $\Gamma(n) = \int_0^\infty t^{n-1}e^{-t} dt$ with $n-1 = 3/2$, so $n = 5/2$:
        $$I = \frac{1}{2} \Gamma\left(\frac{5}{2}\right) = \frac{1}{2} \cdot \frac{3}{2} \cdot \frac{1}{2} \cdot \Gamma\left(\frac{1}{2}\right) = \frac{3}{8}\sqrt{\pi}$$

        \item % 3b
        Let $u = 3x$, then $du = 3 \, dx$, or $dx = \frac{1}{3} du$ and $x = \frac{u}{3}$. \\
        Substituting into the integral:
        $$I = \int_{0}^{\infty} e^{-u} \sqrt{\frac{u}{3}} \frac{1}{3} du = \frac{1}{3\sqrt{3}} \int_{0}^{\infty} u^{1/2} e^{-u} du$$
        This matches the Gamma function form with $n-1 = 1/2$, so $n = 3/2$:
        $$I = \frac{1}{3\sqrt{3}} \Gamma\left(\frac{3}{2}\right) = \frac{1}{3\sqrt{3}} \cdot \frac{1}{2} \cdot \sqrt{\pi} = \frac{\sqrt{\pi}}{6\sqrt{3}} = \frac{\sqrt{3\pi}}{18}$$
    \end{enumerate}
    \item
    \begin{enumerate}
        \item[(b)]
        Assuming definition (7.53) is $B(x, y) = \int_0^1 t^{x-1}(1-t)^{y-1} dt$:
        $$B(x, 1) = \int_0^1 t^{x-1}(1-t)^{1-1} dt = \int_0^1 t^{x-1}(1) dt$$
        Evaluating the integral:
        $$B(x, 1) = \left[ \frac{t^x}{x} \right]_0^1 = \frac{1^x}{x} - \frac{0^x}{x} = \frac{1}{x} = x^{-1}$$

        \item[(c)]
        Using the integral definition:
        $$LHS = \int_0^1 t^x(1-t)^{y-1} dt + \int_0^1 t^{x-1}(1-t)^y dt$$
        Factor out the common term $t^{x-1}(1-t)^{y-1}$:
        $$LHS = \int_0^1 t^{x-1}(1-t)^{y-1} [t + (1-t)] dt$$
        Since $t + 1 - t = 1$:
        $$LHS = \int_0^1 t^{x-1}(1-t)^{y-1} (1) dt = B(x, y) = RHS$$
    \end{enumerate}

    \item % Exercise 6
    The duplication formula is: $\Gamma(2x) = \frac{2^{2x-1}}{\sqrt{\pi}} \Gamma(x) \Gamma(x + \frac{1}{2})$. \\
    Substitute $x$ with $x-1$ in the RHS:
    $$RHS_{x-1} = \frac{2^{2(x-1)-1}}{\sqrt{\pi}} \Gamma(x-1) \Gamma(x-1 + \frac{1}{2}) = \frac{2^{2x-3}}{\sqrt{\pi}} \frac{\Gamma(x)}{x-1} \frac{\Gamma(x+1/2)}{x-1/2}$$
    Combine the constants:
    $$RHS_{x-1} = \frac{2^{2x-1} \cdot 2^{-2}}{\sqrt{\pi} (x-1) \frac{2x-1}{2}} \Gamma(x) \Gamma(x+1/2) = \frac{1}{2(x-1)(2x-1)} \left[ \frac{2^{2x-1}}{\sqrt{\pi}} \Gamma(x) \Gamma(x+1/2) \right]$$
    By the assumption that the formula holds for $x$, the term in brackets is $\Gamma(2x)$:
    $$RHS_{x-1} = \frac{\Gamma(2x)}{2(x-1)(2x-1)} = \frac{\Gamma(2x)}{(2x-2)(2x-1)}$$
    Applying $\Gamma(n) = (n-1)\Gamma(n-1)$ repeatedly: $\Gamma(2x) = (2x-1)\Gamma(2x-1) = (2x-1)(2x-2)\Gamma(2x-2)$.
    $$RHS_{x-1} = \frac{(2x-1)(2x-2)\Gamma(2x-2)}{(2x-2)(2x-1)} = \Gamma(2x-2)$$
    This matches the LHS for $x-1$, proving the step. By induction, since it holds for $x>0$, it extends to all $x$.

    \item % Exercise 7
   
    $I = \frac{1}{2} B\left(\frac{k+1}{2}, \frac{1}{2}\right)$. \\
    Using Theorem 7.55 ($B(x,y) = \frac{\Gamma(x)\Gamma(y)}{\Gamma(x+y)}$):
    $$I = \frac{\Gamma(\frac{k+1}{2})\Gamma(\frac{1}{2})}{2\Gamma(\frac{k}{2} + 1)} = \frac{\sqrt{\pi} \Gamma(\frac{k+1}{2})}{2\Gamma(\frac{k}{2} + 1)}$$
    Case 1: $k$ is even ($k=2n$) \\
    $$I = \frac{\sqrt{\pi} \Gamma(n + \frac{1}{2})}{2\Gamma(n+1)} = \frac{\sqrt{\pi} \cdot \frac{(2n-1)!!}{2^n}\sqrt{\pi}}{2 \cdot n!} = \frac{\pi (2n-1)!!}{2^{n+1} n!}$$
    Case 2: $k$ is odd ($k=2n-1$) \\
    $$I = \frac{\sqrt{\pi} \Gamma(n)}{2\Gamma(n + \frac{1}{2})} = \frac{\sqrt{\pi} (n-1)!}{2 \cdot \frac{(2n-1)!!}{2^n}\sqrt{\pi}} = \frac{2^{n-1}(n-1)!}{(2n-1)!!}$$

    \item % Exercise 9
    Given $I_{\alpha}[f](x) = \frac{1}{\Gamma(\alpha)} \int_{0}^{x} (x - t)^{\alpha - 1} f(t) dt$.
    
    \begin{enumerate}
        \item Show that $\frac{d}{dx} I_{\alpha+1}[f] = I_{\alpha}[f]$. \\
        By definition:
        $$I_{\alpha+1}[f](x) = \frac{1}{\Gamma(\alpha+1)} \int_{0}^{x} (x - t)^{\alpha} f(t) dt$$
        Using the Leibniz Integral Rule for differentiation:
        $$\frac{d}{dx} I_{\alpha+1}[f](x) = \frac{1}{\Gamma(\alpha+1)} \left[ (x-x)^\alpha f(x) + \int_{0}^{x} \frac{\partial}{\partial x} (x - t)^{\alpha} f(t) dt \right]$$
        For $\alpha > 0$, the boundary term $(x-x)^\alpha$ is 0. Using $\Gamma(\alpha+1) = \alpha\Gamma(\alpha)$:
        $$\frac{d}{dx} I_{\alpha+1}[f](x) = \frac{1}{\alpha\Gamma(\alpha)} \int_{0}^{x} \alpha(x - t)^{\alpha-1} f(t) dt = \frac{1}{\Gamma(\alpha)} \int_{0}^{x} (x - t)^{\alpha-1} f(t) dt$$
        This is exactly $I_{\alpha}[f](x)$. For $I_1[f]$, the integral is $\int_0^x f(t) dt$, and its derivative is $f(x)$ by the Fundamental Theorem of Calculus.

        \item Show $I_{\alpha}[I_{\beta}[f]] = I_{\alpha+\beta}[f]$. \\
        $$I_{\alpha}[I_{\beta}[f]](x) = \frac{1}{\Gamma(\alpha)\Gamma(\beta)} \int_{0}^{x} (x-t)^{\alpha-1} \left( \int_{0}^{t} (t-u)^{\beta-1} f(u) du \right) dt$$
        Changing the order of integration (where $0 \le u \le t \le x$):
        $$\frac{1}{\Gamma(\alpha)\Gamma(\beta)} \int_{0}^{x} f(u) \left( \int_{u}^{x} (x-t)^{\alpha-1} (t-u)^{\beta-1} dt \right) du$$
        Let $t = u + s(x-u)$, then $dt = (x-u) ds$. The inner integral becomes: 
        $$(x-u)^{\alpha+\beta-1} \int_{0}^{1} (1-s)^{\alpha-1} s^{\beta-1} ds = (x-u)^{\alpha+\beta-1} B(\beta, \alpha)$$
        Using $B(\beta, \alpha) = \frac{\Gamma(\beta)\Gamma(\alpha)}{\Gamma(\beta+\alpha)}$, the constants cancel to leave:
        $$I_{\alpha}[I_{\beta}[f]](x) = \frac{1}{\Gamma(\alpha+\beta)} \int_{0}^{x} (x-u)^{\alpha+\beta-1} f(u) du = I_{\alpha+\beta}[f](x)$$
    \end{enumerate}

    \item % Exercise 10
    Test the following series for convergence using the Gamma function.
    
    \begin{enumerate}
        \item $\sum_{0}^{\infty} \frac{1 \cdot 4 \dots (3n+1)}{2 \cdot 5 \dots (3n+2)}$. \\
        The general term $a_n$ can be written using Gamma functions:
        $$a_n = \frac{3^n \frac{\Gamma(n + 1 + 1/3)}{\Gamma(1 + 1/3)}}{3^n \frac{\Gamma(n + 1 + 2/3)}{\Gamma(1 + 2/3)}} = C \cdot \frac{\Gamma(n + 4/3)}{\Gamma(n + 5/3)}$$
        $\frac{\Gamma(n+a)}{\Gamma(n+b)} \sim n^{a-b}$ as $n \to \infty$:
        $$a_n \sim n^{(4/3 - 5/3)} = n^{-1/3}$$
        Since $\sum n^{-1/3}$ is a $p$-series with $p = 1/3 \le 1$, the series \textbf{diverges}.

        \item $\sum_{0}^{\infty} \frac{4^n n!}{5 \cdot 9 \dots (4n+5)}$. \\
        
        The denominator is a product of terms in the form $4k+1$ starting from $k=1$. \\
        
        $a_n = \frac{4^n \Gamma(n+1)}{4^{n+1} \frac{\Gamma(n + 1 + 5/4)}{\Gamma(5/4)}} = C' \cdot \frac{\Gamma(n+1)}{\Gamma(n + 9/4)}$
        $$a_n \sim n^{(1 - 9/4)} = n^{-5/4}$$
        Since $\sum n^{-5/4}$ is a $p$-series with $p = 5/4 > 1$, the series \textbf{converges}.
    \end{enumerate}
    
    \item % Exercise 2 (Coin Tossing)
    
    To estimate for large $n$, we apply Stirling's formula, which states that $n! \sim \sqrt{2\pi n} \left(\frac{n}{e}\right)^n$.
    
    \begin{itemize}
        \item First, approximate the numerator: $(2n)! \sim \sqrt{2\pi(2n)} \left(\frac{2n}{e}\right)^{2n} = \sqrt{4\pi n} \frac{2^{2n} n^{2n}}{e^{2n}}$.
        \item Next, approximate the squared factorial in the denominator: $(n!)^2 \sim \left(\sqrt{2\pi n} \frac{n^n}{e^n}\right)^2 = 2\pi n \frac{n^{2n}}{e^{2n}}$.
    \end{itemize}

    Substituting these approximations into the formula for $P_n$:
    $$P_n \sim \frac{\sqrt{4\pi n} \cdot \frac{2^{2n} n^{2n}}{e^{2n}}}{\left(2\pi n \cdot \frac{n^{2n}}{e^{2n}}\right) \cdot 2^{2n}}$$

    Canceling common terms ($\frac{n^{2n}}{e^{2n}}$ and $2^{2n}$):
    $$P_n \sim \frac{\sqrt{4\pi n}}{2\pi n} = \frac{2\sqrt{\pi n}}{2\pi n} = \frac{\sqrt{\pi n}}{\pi n}$$

    Simplifying the final result:
    $$P_n \sim \frac{1}{\sqrt{\pi n}}$$


\item $f(\theta) = \begin{cases} -1 & (-\pi < \theta < 0) \\ 1 & (0 < \theta < \pi) \end{cases}$

Since $f$ is odd, $a_n = 0$ for all $n$. The sine coefficients are
\[
b_n = \frac{1}{\pi}\int_{-\pi}^{\pi} f(\theta)\sin(n\theta)\,d\theta
= \frac{2}{\pi}\int_0^{\pi} \sin(n\theta)\,d\theta
= \frac{2}{\pi}\Bigl[-\tfrac{\cos(n\theta)}{n}\Bigr]_0^{\pi}
= \frac{2(1-(-1)^n)}{n\pi}.
\]
This vanishes for even $n$ and equals $\tfrac{4}{n\pi}$ for odd $n$. Hence
\[
f(\theta) \sim \frac{4}{\pi}\sum_{k=0}^{\infty}\frac{\sin\bigl((2k+1)\theta\bigr)}{2k+1}
= \frac{4}{\pi}\!\left(\sin\theta + \frac{\sin 3\theta}{3} + \frac{\sin 5\theta}{5} + \cdots\right).
\]

\item \textbf{$f(\theta)=|\sin\theta|$}

Since $f$ is even, $b_n = 0$. The constant term is
\[
\frac{a_0}{2} = \frac{1}{\pi}\int_0^{\pi}\sin\theta\,d\theta = \frac{2}{\pi}.
\]
Using the hint $\sin a\cos b = \tfrac{1}{2}[\sin(a+b)+\sin(a-b)]$,
\[
a_n = \frac{2}{\pi}\int_0^{\pi}\sin\theta\cos(n\theta)\,d\theta
= \frac{1}{\pi}\int_0^{\pi}\bigl[\sin((1+n)\theta)+\sin((1-n)\theta)\bigr]d\theta.
\]
Evaluating for $n\neq 1$ gives $a_n = \tfrac{-4}{\pi(n^2-1)}$ when $n$ is even and $a_n=0$ when $n$ is odd; one checks separately that $a_1=0$. Therefore
\[
f(\theta) \sim \frac{2}{\pi} - \frac{4}{\pi}\sum_{k=1}^{\infty}\frac{\cos(2k\theta)}{4k^2-1}
= \frac{2}{\pi} - \frac{4}{\pi}\!\left(\frac{\cos 2\theta}{3}+\frac{\cos 4\theta}{15}+\frac{\cos 6\theta}{35}+\cdots\right).
\]

\item \textbf{$f(\theta)=\theta^2$}

Since $f$ is even, $b_n=0$. The constant term is
\[
\frac{a_0}{2} = \frac{1}{\pi}\int_0^{\pi}\theta^2\,d\theta = \frac{\pi^2}{3}.
\]
Integrating by parts twice:
\[
a_n = \frac{2}{\pi}\int_0^{\pi}\theta^2\cos(n\theta)\,d\theta
= -\frac{4}{n\pi}\int_0^{\pi}\theta\sin(n\theta)\,d\theta
= -\frac{4}{n\pi}\cdot\frac{(-1)^{n+1}\pi}{n}
= \frac{4(-1)^n}{n^2}.
\]
Hence
\[
f(\theta) \sim \frac{\pi^2}{3}+4\sum_{n=1}^{\infty}\frac{(-1)^n}{n^2}\cos(n\theta)
= \frac{\pi^2}{3} - 4\cos\theta + \cos 2\theta - \frac{4}{9}\cos 3\theta + \cdots
\]

\item \textbf{$f(\theta)=\theta(\pi-|\theta|)$}

Since $f(-\theta)=(-\theta)(\pi-|\theta|)=-f(\theta)$, $f$ is odd and $a_n=0$. Using the results
\[
\int_0^\pi\theta\sin(n\theta)\,d\theta = \frac{(-1)^{n+1}\pi}{n},
\qquad
\int_0^\pi\theta^2\sin(n\theta)\,d\theta = \frac{(-1)^{n+1}\pi^2}{n}+\frac{2(1-(-1)^n)}{n^3},
\]
we get
\[
b_n = \frac{2}{\pi}\!\left[\pi\cdot\frac{(-1)^{n+1}\pi}{n}
- \frac{(-1)^{n+1}\pi^2}{n} - \frac{2(1-(-1)^n)}{n^3}\right]
= \frac{-4(1-(-1)^n)}{n^3\pi}.
\]
This vanishes for even $n$ and equals $-\tfrac{8}{n^3\pi}$ for odd $n$. Hence
\[
f(\theta) \sim -\frac{8}{\pi}\sum_{k=0}^{\infty}\frac{\sin\bigl((2k+1)\theta\bigr)}{(2k+1)^3}
= -\frac{8}{\pi}\!\left(\sin\theta + \frac{\sin 3\theta}{27}+\frac{\sin 5\theta}{125}+\cdots\right).
\]



    

\end{enumerate}



\end{document}
