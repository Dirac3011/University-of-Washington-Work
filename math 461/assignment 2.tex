\documentclass[12pt]{article}
\usepackage{bigints}
\usepackage{graphicx}			% Use this package to include images
\usepackage{amsmath}	
\usepackage{amssymb}
\usepackage{amsfonts}
\usepackage{polynom}
\usepackage{listings}
% A library of many standard math expressions
\graphicspath{ {./Images/} }
\usepackage[margin=1in]{geometry}% Sets 1in margins. 
\newcommand{\qed}[0]{$\blacksquare$}
\usepackage{fancyhdr}			% Creates headers and footers
\usepackage{enumerate}          %These two package give custom labels to a list
\usepackage[shortlabels]{enumitem}


% Creates the header and footer. You can adjust the look and feel of these here.
\pagestyle{fancy}
\fancyhead[l]{Aditya Gupta}
\fancyhead[c]{Math 461 Homework \#2}
\fancyhead[r]{\today}
\fancyfoot[c]{\thepage}
\renewcommand{\headrulewidth}{0.2pt} %Creates a horizontal line underneath the header
\setlength{\headheight}{15pt} %Sets enough space for the header
\begin{document}

\begin{enumerate}
\item
Let the five points have integer coordinates \((x_i,y_i)\) for \(i=1,\dots,5\).
Consider the parity of each point modulo \(2\).  
Each point determines a pair \((x_i \bmod 2,\, y_i \bmod 2)\), and there are only \(4\) possible such classes. 
    By the Pigeonhole Principle, two of the points, say \((x_1,y_1)\) and \((x_2,y_2)\), lie in the same class, so \(x_1 \equiv x_2 \pmod 2\) and \(y_1 \equiv y_2 \pmod 2\).  
Hence \(x_1+x_2\) and \(y_1+y_2\) are even, and the midpoint
\[
\left(\frac{x_1+x_2}{2},\,\frac{y_1+y_2}{2}\right)
\]
has integer coordinates.

\item
Partition the equilateral triangle of side length \(1\) into \(16\) smaller equilateral triangles of side length \(1/4\). We do this by joining the midpoints of the sides. This forms 4 triangles. Then each of those 4 triangles are again divided by the same method giving us 16 triangles.
These \(16\) regions are the pigeonholes, and the \(17\) given points are the pigeons.  
By the Pigeonhole Principle, two points lie in the same small triangle.  
The maximum possible distance between two points in one such triangle is at most its side length \(1/4\).  
Hence there exist two points whose distance is no more than \(1/4\).

\item
We perform the modulo 7 on each of the \(64\) randomly selected integers.  
There are \(7\) possible remainders, which are the pigeonholes, and the integers are the pigeons.  
By the Pigeonhole Principle, some remainder class contains at least
\(\lceil 64/7 \rceil = 10\) integers.  
Any two integers in this class differ by a multiple of \(7\), so these ten integers satisfy the required property.

\item
Partition the set \(\{1,2,\dots,100\}\) into the \(50\) pairs
\[
\{1,2\},\{3,4\},\dots,\{99,100\}.
\]
These pairs are the pigeonholes.  
Selecting \(51\) numbers places \(51\) pigeons into \(50\) holes, so one pair contains two chosen numbers.  
Those two numbers are consecutive.

\item
Partition \(\{1,2,\dots,100\}\) into the \(50\) pairs
\[
\{1,100\},\{2,99\},\dots,\{50,51\}.
\]
Each pair sums to \(101\).  
Choosing \(51\) numbers forces two of them to lie in the same pair, so their sum is \(101\).

\item
The first digit has \(9\) choices (\(1\) through \(9\)).  
The second digit then has \(9\) choices (any digit except the first), the third has \(8\) choices, and the fourth has \(7\) choices.  
Thus the number of such integers is
\[
9 \cdot 9 \cdot 8 \cdot 7 = 4536.
\]

\item
Write a four-digit number as \(abcd\), where \(a \in \{1,\dots,9\}\) and \(b,c,d \in \{0,\dots,9\}\).
The parity of the digit sum \(a+b+c+d\) depends on how many of the digits are odd.

There are \(5\) even digits and \(5\) odd digits among \(\{0,\dots,9\}\).
For the first digit, there are \(4\) even choices (\(2,4,6,8\)) and \(5\) odd choices (\(1,3,5,7,9\)).
For each of the remaining three digits, there are \(5\) even and \(5\) odd choices.

The digit sum is even exactly when an even number of the four digits are odd.
We count these cases.

If all four digits are even, the number of choices is
\[
4 \cdot 5^3.
\]
If exactly two digits are odd, choose which two of the four positions are odd and count choices:
\[
\binom{4}{2} \cdot 5^2 \cdot 5^2,
\]
except that the first digit must be chosen from the \(5\) odd digits if it is odd, or from the \(4\) even digits if it is even.
Taking this into account, the total number of four-digit integers with even digit sum is
\[
4 \cdot 5^3 + \binom{4}{2} \cdot 5^4 = 4500.
\]
Thus there are \(4500\) four-digit positive integers whose digit sum is even.


\item
The largest possible number is \(2^{n-1}\).  
So we fix an element \(x \in [n]\) and take all subsets of \([n]\) that contain \(x\); there are \(2^{n-1}\) such subsets, and any two intersect in at least \(x\).  
Now suppose we select more than \(2^{n-1}\) subsets.  
Pair each subset \(A \subseteq [n]\) with its complement \(A^c\).  
There are \(2^{n-1}\) such complementary pairs.  
By the Pigeonhole Principle, selecting \(2^{n-1}+1\) subsets forces us to choose both \(A\) and \(A^c\) for some pair, but these two sets are disjoint, a contradiction.  
Hence the maximum is \(2^{n-1}\).

\item
For each element \(k \in \{1,\dots,n\}\), there are three possibilities:  
\(k \notin B\), \(k \in B \setminus A\), or \(k \in A\).  
Each choice produces a unique ordered pair \((A,B)\) with \(A \subseteq B\).  
Thus the total number of ordered pairs is $3^n$
\end{enumerate}

\end{document}
