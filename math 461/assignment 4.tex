\documentclass[12pt]{article}
\usepackage{bigints}
\usepackage{graphicx}			% Use this package to include images
\usepackage{amsmath}	
\usepackage{amssymb}
\usepackage{amsfonts}
\usepackage{polynom}
\usepackage{listings}
% A library of many standard math expressions
\graphicspath{ {./Images/} }
\usepackage[margin=1in]{geometry}% Sets 1in margins. 
\newcommand{\qed}[0]{$\blacksquare$}
\usepackage{fancyhdr}			% Creates headers and footers
\usepackage{enumerate}          %These two package give custom labels to a list
\usepackage[shortlabels]{enumitem}


% Creates the header and footer. You can adjust the look and feel of these here.
\pagestyle{fancy}
\fancyhead[l]{Aditya Gupta}
\fancyhead[c]{Math 461 Homework \#4}
\fancyhead[r]{\today}
\fancyfoot[c]{\thepage}
\renewcommand{\headrulewidth}{0.2pt} %Creates a horizontal line underneath the header
\setlength{\headheight}{15pt} %Sets enough space for the header
\begin{document}
\begin{enumerate}
\item We use the Inclusion Exclusion principle. Let $S$ be the set $\{1, 2, \dots, 1000\}$.

Let $A$ be the set of perfect squares in $S$. Since $31^2 = 961$ and $32^2 = 1024$, we have:
$$|A| = \lfloor \sqrt{1000} \rfloor = 31$$

Let $B$ be the set of perfect cubes in $S$. Since $10^3 = 1000$, we have:
$$|B| = \lfloor \sqrt[3]{1000} \rfloor = 10$$

The intersection $A \cap B$ consists of numbers that are both squares and cubes, which are perfect sixth powers. Since $3^6 = 729$ and $4^6 = 4096$, we have:
$$|A \cap B| = \lfloor \sqrt[6]{1000} \rfloor = 3$$

The number of integers that are either squares or cubes is:
$$|A \cup B| = |A| + |B| - |A \cap B| = 31 + 10 - 3 = 38$$

The number of integers that are neither is:
$$1000 - 38 = 962$$

\item To find the number of functions $f: \{1, 2, 3, 4, 5\} \to \{1, 2, 3, 4, 5\}$ such that $f(1) = f(2)$ or $f(1) = f(3)$, let $P_1$ be the property that $f(1) = f(2)$ and $P_2$ be the property that $f(1) = f(3)$.

The number of functions satisfying $P_1$:
There are $5$ choices for the value of $f(1)$, which determines $f(2)$. The values $f(3), f(4), f(5)$ can each be any of the $5$ values.
$$|P_1| = 5 \times 1 \times 5 \times 5 \times 5 = 5^4 = 625$$

The number of functions satisfying $P_2$:
By symmetry, there are $5$ choices for $f(1)$ (which determines $f(3)$) and $5$ choices for each of the other three inputs.
$$|P_2| = 5 \times 5 \times 1 \times 5 \times 5 = 5^4 = 625$$

The number of functions satisfying both $P_1$ and $P_2$:
This requires $f(1) = f(2) = f(3)$. There are $5$ choices for this common value, and $5$ choices for $f(4)$ and $f(5)$ each.
$$|P_1 \cap P_2| = 5 \times 1 \times 1 \times 5 \times 5 = 5^3 = 125$$

By the Principle of Inclusion-Exclusion, the total number of functions is:
$$|P_1 \cup P_2| = |P_1| + |P_2| - |P_1 \cap P_2| = 625 + 625 - 125 = 1125$$

\item Let $D$, $C$, and $B$ represent the sets of residents who own a dog, a cat, and a bird, respectively. The formula for the union of three sets via inclusion exclusion principle is:
$$|D \cup C \cup B| = |D| + |C| + |B| - (|D \cap C| + |D \cap B| + |C \cap B|) + |D \cap C \cap B|$$

We are given the following values:
\begin{itemize}
    \item Total residents $|D \cup C \cup B| = 30$
    \item Dog owners $|D| = 20$
    \item Cat owners $|C| = 15$
    \item Bird owners $|B| = 8$
    \item Dog and Cat $|D \cap C| = 8$
    \item Dog and Bird $|D \cap B| = 4$
    \item Cat and Bird $|C \cap B| = 3$
\end{itemize}

Plugging these into the formula:
$$30 = 20 + 15 + 8 - (8 + 4 + 3) + |D \cap C \cap B|$$
$$30 = 43 - 15 + |D \cap C \cap B|$$
$$30 = 28 + |D \cap C \cap B|$$
$$|D \cap C \cap B| = 2$$

So, 2 residents own all three animals.

\item We use the Principle of Inclusion-Exclusion to solve this question

Firstly, the total number of permutations of the word ESTATE is:
$$\frac{6!}{2!2!} = \frac{720}{4} = 180$$

Let $P_E$ be the property that the two E's are consecutive, and $P_T$ be the property that the two T's are consecutive.

a. For $|P_E|$: We treat (EE) as a single block. We then have 5 items: (EE), S, T, A, T.
$$\frac{5!}{2!} = \frac{120}{2} = 60$$

b. For $|P_T|$: We treat (TT) as a single block. We then have 5 items: E, S, (TT), A, E.
$$\frac{5!}{2!} = \frac{120}{2} = 60$$

c. For $|P_E \cap P_T|$: Treat (EE) and (TT) as single blocks. We then have 4 items: (EE), S, (TT), A.
$$4! = 24$$

By inclusion exclusion principle, the number of arrangements with at least one pair of consecutive identical letters is:
$$|P_E \cup P_T| = |P_E| + |P_T| - |P_E \cap P_T| = 60 + 60 - 24 = 96$$

The number of arrangements where no two consecutive letters are the same is:
$$180 - 96 = 84$$
\item To find the number of weak compositions of $8$ into $4$ parts ($x_1 + x_2 + x_3 + x_4 = 8$, where $x_i \ge 0$) such that no part is equal to $2$, we use the Principle of Inclusion-Exclusion.

Let $S$ be the set of all weak compositions. The total number is:
$$\binom{8 + 4 - 1}{4 - 1} = \binom{11}{3} = 165$$

Let $P_i$ be the property that $x_i = 2$. We want to find the number of compositions that satisfy none of these properties.

1. For $|P_i|$: Fix one $x_i = 2$. The remaining 3 parts must sum to $6$.
$$\binom{3}{1} \times \binom{6 + 3 - 1}{3 - 1} = 4 \times \binom{8}{2} = 4 \times 28 = 112$$

2. For $|P_i \cap P_j|$: Fix two parts equal to $2$. The remaining 2 parts must sum to $4$.
$$\binom{4}{2} \times \binom{4 + 2 - 1}{2 - 1} = 6 \times \binom{5}{1} = 6 \times 5 = 30$$

3. For $|P_i \cap P_j \cap P_k|$: Fix three parts equal to $2$. The remaining 1 part must sum to $2$. (Note: This forces the 4th part to also be 2).
$$\binom{4}{3} \times \binom{2 + 1 - 1}{1 - 1} = 4 \times 1 = 4$$

4. For $|P_1 \cap P_2 \cap P_3 \cap P_4|$: All four parts are $2$:
$$\binom{4}{4} \times 1 = 1$$

Using inclusion exclusion principle:
$$165 - (112) + (30) - (4) + (1) = 80$$

\item To find the number of strings of length $12$ using letters $\{A, B, C, D\}$ where each letter appears at least once:

Let $U$ be the set of all strings of length $12$ using 4 letters. $|U| = 4^{12}$.
Let $P_A, P_B, P_C, P_D$ be the properties that the letters $A, B, C, D$ (respectively) do NOT appear in the string.

1. Strings missing at least one specific letter: $\binom{4}{1} \times 3^{12}$

2. Strings missing at least two specific letters: $\binom{4}{2} \times 2^{12}$

3. Strings missing at least three specific letters: $\binom{4}{3} \times 1^{12}$

4. Strings missing all four letters: $\binom{4}{4} \times 0^{12} = 0$

By the Principle of Inclusion-Exclusion, the number of strings containing all letters is:
$$4^{12} - \binom{4}{1}3^{12} + \binom{4}{2}2^{12} - \binom{4}{3}1^{12}$$
$$16,777,216 - 4(531,441) + 6(4,096) - 4(1)$$
$$16,777,216 - 2,125,764 + 24,576 - 4 = 14,676,024$$

\item We need to determine the number of solutions to $x_1 + x_2 + \dots + x_{10} = 100$ in positive integers such that $1 \le x_i \le 30$.

First, let $y_i = x_i - 1$, so $y_i \ge 0$. The equation becomes:
$$(y_1 + 1) + (y_2 + 1) + \dots + (y_{10} + 1) = 100 \implies y_1 + y_2 + \dots + y_{10} = 90$$
The number of solutions in positive integers with no upper bound is:
$$\binom{90 + 10 - 1}{10 - 1} = \binom{99}{9}$$

Now, let $P_i$ be the property that $x_i > 30$. This is equivalent to $x_i \ge 31$.
To find the number of solutions where at least one $x_j \in S$ satisfies $x_j \ge 31$, we substitute $x_j = z_j + 30$ (where $z_j \ge 1$) for each $j \in S$. For the remaining $i \notin S$, we keep $x_i \ge 1$.

If $|S| = k$, the sum becomes:
$$\sum_{j \in S} (z_j + 30) + \sum_{i \notin S} x_i = 100 \implies \sum z_j + \sum x_i = 100 - 30k$$
The number of solutions for a specific subset $S$ of size $k$ is:
$$\binom{(100 - 30k) - 1}{10 - 1} = \binom{99 - 30k}{9}$$

We apply the inclusion exclusion principle. Note that if $k \ge 4$, $100 - 30(4) = -20$, which is impossible. Thus, we only consider $k=1, 2, 3$.

The number of solutions where at least one $x_i > 30$ is:
$$S_1 - S_2 + S_3 = \binom{10}{1}\binom{69}{9} - \binom{10}{2}\binom{39}{9} + \binom{10}{3}\binom{9}{9}$$

The number of solutions satisfying $1 \le x_i \le 30$ is the total minus the restricted sets:
$$\binom{99}{9} - \left[ \binom{10}{1}\binom{69}{9} - \binom{10}{2}\binom{39}{9} + \binom{10}{3}\binom{9}{9} \right]$$
$$\binom{99}{9} - 10\binom{69}{9} + 45\binom{39}{9} - 120(1)$$
\end{enumerate}


\end{document}
