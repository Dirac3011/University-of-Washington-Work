\documentclass[12pt]{article}
\usepackage{bigints}
\usepackage{graphicx}			% Use this package to include images
\usepackage{amsmath}	
\usepackage{amssymb}
\usepackage{amsfonts}
\usepackage{polynom}
\usepackage{listings}
% A library of many standard math expressions
\graphicspath{ {./Images/} }
\usepackage[margin=1in]{geometry}% Sets 1in margins. 
\newcommand{\qed}[0]{$\blacksquare$}
\usepackage{fancyhdr}			% Creates headers and footers
\usepackage{enumerate}          %These two package give custom labels to a list
\usepackage[shortlabels]{enumitem}


% Creates the header and footer. You can adjust the look and feel of these here.
\pagestyle{fancy}
\fancyhead[l]{Aditya Gupta}
\fancyhead[c]{Math 461 Homework \#3}
\fancyhead[r]{\today}
\fancyfoot[c]{\thepage}
\renewcommand{\headrulewidth}{0.2pt} %Creates a horizontal line underneath the header
\setlength{\headheight}{15pt} %Sets enough space for the header
\begin{document}
\begin{enumerate}

\item
There are 4 choices for size and 3 choices for crust.  
For each of the 10 toppings, the customer may either include it or not, giving $2^{10}$ possible topping selections.

\[
4 \cdot 3 \cdot 2^{10} = 12 \cdot 1024 = 12288
\]

So, $12288$ different pizzas can be made.

\item
Label the sophomores $S_1,\dots,S_n$, the juniors $J_1,\dots,J_n$, and the seniors $R_1,\dots,R_n$.

First, pair each sophomore with a junior.  
This is equivalent to choosing a bijection from the set of sophomores to the set of juniors, which can be done in $n!$ ways.

Now each sophomore--junior pair must be matched with exactly one senior.  
Again, this is choosing a bijection between the $n$ existing pairs and the $n$ seniors, which can be done in $n!$ ways.

Each final group consists of exactly one sophomore, one junior, and one senior, and the order of the groups does not matter because the construction produces unordered triples.

Therefore, the total number of such divisions is
\[
n! \cdot n! = (n!)^2.
\]


\item
(a) BANANAS has 7 letters: B(1), A(3), N(2), S(1).

\[
\frac{7!}{3!2!}=420
\]

(b) Total rearrangements of APPLES:

\[
\frac{6!}{2!}=360
\]

Subtract cases where first two letters are the same. Only P can repeat.

Fix PP at the front; remaining letters A, L, E, S can be arranged in $4!$ ways.

\[
360-24=336
\]

\item
$[7]=\{1,2,3,4,5,6,7\}$ and $[3]=\{1,2,3\}$.

We want 5-element subsets of $[7]$ that do not contain all of $\{1,2,3\}$.

Total 5-subsets of $[7]$:

\[
{7 \choose 5}=21
\]

Subsets containing $\{1,2,3\}$: choose 2 more from remaining $\{4,5,6,7\}$.

\[
{4 \choose 2}=6
\]

Therefore,

\[
21-6=15
\]

\item
The choices for appetizers, main dishes, and desserts are independent.

First choose 3 different appetizers from the 6 available:
\[
{6 \choose 3}.
\]

Next choose 3 different main dishes from the 5 available:
\[
{5 \choose 3}.
\]

Finally choose 2 different desserts from the 3 available:
\[
{3 \choose 2}.
\]

Since these choices are independent, we multiply:
\[
{6 \choose 3}{5 \choose 3}{3 \choose 2}
= 20 \cdot 10 \cdot 3
= 600.
\]

Thus, there are $600$ possible orders.


\item
Let $U=\{1,2,\dots,10\}$. The sum of all elements of $U$ is
\[
1+2+\cdots+10 = 55.
\]

Define a function $\phi$ from subsets of $U$ to subsets of $U$ by
\[
\phi(A)=U\setminus A.
\]

If a subset $A$ has sum $20$, then its complement has sum
\[
55-20=35.
\]

The function $\phi$ is a bijection because:
\begin{itemize}
\item every subset has a unique complement,
\item taking the complement twice returns the original set.
\end{itemize}

Thus, $\phi$ gives a one-to-one correspondence between subsets with sum $20$ and subsets with sum $35$.

Therefore,
\[
f(20)=f(35).
\]


\item
Consider all subsets of $\{1,2,\dots,2n\}$.  
There are $2^{2n}=4^n$ such subsets.

The number ${2n \choose n}$ counts the subsets of size exactly $n$.

Since not all subsets have size $n$, the number of $n$-element subsets must be strictly less than the total number of subsets. Therefore,
\[
{2n \choose n} < 2^{2n} = 4^n.
\]

This proves that for all $n \ge 1$,
\[
{2n \choose n} < 4^n.
\]


\item
We count in two ways.

Left side:  
Choose $k$ people from $n$, then choose ordered president and vice-president from those $k$:

\[
\sum_{k=2}^n k(k-1){n \choose k}
\]

Right side:  
Choose president and vice-president first: $n(n-1)$ ways.  
From remaining $n-2$ people, choose any subset to join the committee:

\[
2^{n-2}
\]

Total:

\[
n(n-1)2^{n-2}
\]

Since both expressions count the same thing,

\[
\sum_{k=2}^n k(k-1){n \choose k}=n(n-1)2^{n-2}
\]

\end{enumerate}


\end{document}
